
% Default to the notebook output style

    


% Inherit from the specified cell style.




    
\documentclass[11pt]{article}

    
    
    \usepackage[T1]{fontenc}
    % Nicer default font (+ math font) than Computer Modern for most use cases
    \usepackage{mathpazo}

    % Basic figure setup, for now with no caption control since it's done
    % automatically by Pandoc (which extracts ![](path) syntax from Markdown).
    \usepackage{graphicx}
    % We will generate all images so they have a width \maxwidth. This means
    % that they will get their normal width if they fit onto the page, but
    % are scaled down if they would overflow the margins.
    \makeatletter
    \def\maxwidth{\ifdim\Gin@nat@width>\linewidth\linewidth
    \else\Gin@nat@width\fi}
    \makeatother
    \let\Oldincludegraphics\includegraphics
    % Set max figure width to be 80% of text width, for now hardcoded.
    \renewcommand{\includegraphics}[1]{\Oldincludegraphics[width=.8\maxwidth]{#1}}
    % Ensure that by default, figures have no caption (until we provide a
    % proper Figure object with a Caption API and a way to capture that
    % in the conversion process - todo).
    \usepackage{caption}
    \DeclareCaptionLabelFormat{nolabel}{}
    \captionsetup{labelformat=nolabel}

    \usepackage{adjustbox} % Used to constrain images to a maximum size 
    \usepackage{xcolor} % Allow colors to be defined
    \usepackage{enumerate} % Needed for markdown enumerations to work
    \usepackage{geometry} % Used to adjust the document margins
    \usepackage{amsmath} % Equations
    \usepackage{amssymb} % Equations
    \usepackage{textcomp} % defines textquotesingle
    % Hack from http://tex.stackexchange.com/a/47451/13684:
    \AtBeginDocument{%
        \def\PYZsq{\textquotesingle}% Upright quotes in Pygmentized code
    }
    \usepackage{upquote} % Upright quotes for verbatim code
    \usepackage{eurosym} % defines \euro
    \usepackage[mathletters]{ucs} % Extended unicode (utf-8) support
    \usepackage[utf8x]{inputenc} % Allow utf-8 characters in the tex document
    \usepackage{fancyvrb} % verbatim replacement that allows latex
    \usepackage{grffile} % extends the file name processing of package graphics 
                         % to support a larger range 
    % The hyperref package gives us a pdf with properly built
    % internal navigation ('pdf bookmarks' for the table of contents,
    % internal cross-reference links, web links for URLs, etc.)
    \usepackage{hyperref}
    \usepackage{longtable} % longtable support required by pandoc >1.10
    \usepackage{booktabs}  % table support for pandoc > 1.12.2
    \usepackage[inline]{enumitem} % IRkernel/repr support (it uses the enumerate* environment)
    \usepackage[normalem]{ulem} % ulem is needed to support strikethroughs (\sout)
                                % normalem makes italics be italics, not underlines
    

    
    
    % Colors for the hyperref package
    \definecolor{urlcolor}{rgb}{0,.145,.698}
    \definecolor{linkcolor}{rgb}{.71,0.21,0.01}
    \definecolor{citecolor}{rgb}{.12,.54,.11}

    % ANSI colors
    \definecolor{ansi-black}{HTML}{3E424D}
    \definecolor{ansi-black-intense}{HTML}{282C36}
    \definecolor{ansi-red}{HTML}{E75C58}
    \definecolor{ansi-red-intense}{HTML}{B22B31}
    \definecolor{ansi-green}{HTML}{00A250}
    \definecolor{ansi-green-intense}{HTML}{007427}
    \definecolor{ansi-yellow}{HTML}{DDB62B}
    \definecolor{ansi-yellow-intense}{HTML}{B27D12}
    \definecolor{ansi-blue}{HTML}{208FFB}
    \definecolor{ansi-blue-intense}{HTML}{0065CA}
    \definecolor{ansi-magenta}{HTML}{D160C4}
    \definecolor{ansi-magenta-intense}{HTML}{A03196}
    \definecolor{ansi-cyan}{HTML}{60C6C8}
    \definecolor{ansi-cyan-intense}{HTML}{258F8F}
    \definecolor{ansi-white}{HTML}{C5C1B4}
    \definecolor{ansi-white-intense}{HTML}{A1A6B2}

    % commands and environments needed by pandoc snippets
    % extracted from the output of `pandoc -s`
    \providecommand{\tightlist}{%
      \setlength{\itemsep}{0pt}\setlength{\parskip}{0pt}}
    \DefineVerbatimEnvironment{Highlighting}{Verbatim}{commandchars=\\\{\}}
    % Add ',fontsize=\small' for more characters per line
    \newenvironment{Shaded}{}{}
    \newcommand{\KeywordTok}[1]{\textcolor[rgb]{0.00,0.44,0.13}{\textbf{{#1}}}}
    \newcommand{\DataTypeTok}[1]{\textcolor[rgb]{0.56,0.13,0.00}{{#1}}}
    \newcommand{\DecValTok}[1]{\textcolor[rgb]{0.25,0.63,0.44}{{#1}}}
    \newcommand{\BaseNTok}[1]{\textcolor[rgb]{0.25,0.63,0.44}{{#1}}}
    \newcommand{\FloatTok}[1]{\textcolor[rgb]{0.25,0.63,0.44}{{#1}}}
    \newcommand{\CharTok}[1]{\textcolor[rgb]{0.25,0.44,0.63}{{#1}}}
    \newcommand{\StringTok}[1]{\textcolor[rgb]{0.25,0.44,0.63}{{#1}}}
    \newcommand{\CommentTok}[1]{\textcolor[rgb]{0.38,0.63,0.69}{\textit{{#1}}}}
    \newcommand{\OtherTok}[1]{\textcolor[rgb]{0.00,0.44,0.13}{{#1}}}
    \newcommand{\AlertTok}[1]{\textcolor[rgb]{1.00,0.00,0.00}{\textbf{{#1}}}}
    \newcommand{\FunctionTok}[1]{\textcolor[rgb]{0.02,0.16,0.49}{{#1}}}
    \newcommand{\RegionMarkerTok}[1]{{#1}}
    \newcommand{\ErrorTok}[1]{\textcolor[rgb]{1.00,0.00,0.00}{\textbf{{#1}}}}
    \newcommand{\NormalTok}[1]{{#1}}
    
    % Additional commands for more recent versions of Pandoc
    \newcommand{\ConstantTok}[1]{\textcolor[rgb]{0.53,0.00,0.00}{{#1}}}
    \newcommand{\SpecialCharTok}[1]{\textcolor[rgb]{0.25,0.44,0.63}{{#1}}}
    \newcommand{\VerbatimStringTok}[1]{\textcolor[rgb]{0.25,0.44,0.63}{{#1}}}
    \newcommand{\SpecialStringTok}[1]{\textcolor[rgb]{0.73,0.40,0.53}{{#1}}}
    \newcommand{\ImportTok}[1]{{#1}}
    \newcommand{\DocumentationTok}[1]{\textcolor[rgb]{0.73,0.13,0.13}{\textit{{#1}}}}
    \newcommand{\AnnotationTok}[1]{\textcolor[rgb]{0.38,0.63,0.69}{\textbf{\textit{{#1}}}}}
    \newcommand{\CommentVarTok}[1]{\textcolor[rgb]{0.38,0.63,0.69}{\textbf{\textit{{#1}}}}}
    \newcommand{\VariableTok}[1]{\textcolor[rgb]{0.10,0.09,0.49}{{#1}}}
    \newcommand{\ControlFlowTok}[1]{\textcolor[rgb]{0.00,0.44,0.13}{\textbf{{#1}}}}
    \newcommand{\OperatorTok}[1]{\textcolor[rgb]{0.40,0.40,0.40}{{#1}}}
    \newcommand{\BuiltInTok}[1]{{#1}}
    \newcommand{\ExtensionTok}[1]{{#1}}
    \newcommand{\PreprocessorTok}[1]{\textcolor[rgb]{0.74,0.48,0.00}{{#1}}}
    \newcommand{\AttributeTok}[1]{\textcolor[rgb]{0.49,0.56,0.16}{{#1}}}
    \newcommand{\InformationTok}[1]{\textcolor[rgb]{0.38,0.63,0.69}{\textbf{\textit{{#1}}}}}
    \newcommand{\WarningTok}[1]{\textcolor[rgb]{0.38,0.63,0.69}{\textbf{\textit{{#1}}}}}
    
    
    % Define a nice break command that doesn't care if a line doesn't already
    % exist.
    \def\br{\hspace*{\fill} \\* }
    % Math Jax compatability definitions
    \def\gt{>}
    \def\lt{<}
    % Document parameters
    \title{intro\_to\_python\_students\_with\_solutions}
    
    
    

    % Pygments definitions
    
\makeatletter
\def\PY@reset{\let\PY@it=\relax \let\PY@bf=\relax%
    \let\PY@ul=\relax \let\PY@tc=\relax%
    \let\PY@bc=\relax \let\PY@ff=\relax}
\def\PY@tok#1{\csname PY@tok@#1\endcsname}
\def\PY@toks#1+{\ifx\relax#1\empty\else%
    \PY@tok{#1}\expandafter\PY@toks\fi}
\def\PY@do#1{\PY@bc{\PY@tc{\PY@ul{%
    \PY@it{\PY@bf{\PY@ff{#1}}}}}}}
\def\PY#1#2{\PY@reset\PY@toks#1+\relax+\PY@do{#2}}

\expandafter\def\csname PY@tok@w\endcsname{\def\PY@tc##1{\textcolor[rgb]{0.73,0.73,0.73}{##1}}}
\expandafter\def\csname PY@tok@c\endcsname{\let\PY@it=\textit\def\PY@tc##1{\textcolor[rgb]{0.25,0.50,0.50}{##1}}}
\expandafter\def\csname PY@tok@cp\endcsname{\def\PY@tc##1{\textcolor[rgb]{0.74,0.48,0.00}{##1}}}
\expandafter\def\csname PY@tok@k\endcsname{\let\PY@bf=\textbf\def\PY@tc##1{\textcolor[rgb]{0.00,0.50,0.00}{##1}}}
\expandafter\def\csname PY@tok@kp\endcsname{\def\PY@tc##1{\textcolor[rgb]{0.00,0.50,0.00}{##1}}}
\expandafter\def\csname PY@tok@kt\endcsname{\def\PY@tc##1{\textcolor[rgb]{0.69,0.00,0.25}{##1}}}
\expandafter\def\csname PY@tok@o\endcsname{\def\PY@tc##1{\textcolor[rgb]{0.40,0.40,0.40}{##1}}}
\expandafter\def\csname PY@tok@ow\endcsname{\let\PY@bf=\textbf\def\PY@tc##1{\textcolor[rgb]{0.67,0.13,1.00}{##1}}}
\expandafter\def\csname PY@tok@nb\endcsname{\def\PY@tc##1{\textcolor[rgb]{0.00,0.50,0.00}{##1}}}
\expandafter\def\csname PY@tok@nf\endcsname{\def\PY@tc##1{\textcolor[rgb]{0.00,0.00,1.00}{##1}}}
\expandafter\def\csname PY@tok@nc\endcsname{\let\PY@bf=\textbf\def\PY@tc##1{\textcolor[rgb]{0.00,0.00,1.00}{##1}}}
\expandafter\def\csname PY@tok@nn\endcsname{\let\PY@bf=\textbf\def\PY@tc##1{\textcolor[rgb]{0.00,0.00,1.00}{##1}}}
\expandafter\def\csname PY@tok@ne\endcsname{\let\PY@bf=\textbf\def\PY@tc##1{\textcolor[rgb]{0.82,0.25,0.23}{##1}}}
\expandafter\def\csname PY@tok@nv\endcsname{\def\PY@tc##1{\textcolor[rgb]{0.10,0.09,0.49}{##1}}}
\expandafter\def\csname PY@tok@no\endcsname{\def\PY@tc##1{\textcolor[rgb]{0.53,0.00,0.00}{##1}}}
\expandafter\def\csname PY@tok@nl\endcsname{\def\PY@tc##1{\textcolor[rgb]{0.63,0.63,0.00}{##1}}}
\expandafter\def\csname PY@tok@ni\endcsname{\let\PY@bf=\textbf\def\PY@tc##1{\textcolor[rgb]{0.60,0.60,0.60}{##1}}}
\expandafter\def\csname PY@tok@na\endcsname{\def\PY@tc##1{\textcolor[rgb]{0.49,0.56,0.16}{##1}}}
\expandafter\def\csname PY@tok@nt\endcsname{\let\PY@bf=\textbf\def\PY@tc##1{\textcolor[rgb]{0.00,0.50,0.00}{##1}}}
\expandafter\def\csname PY@tok@nd\endcsname{\def\PY@tc##1{\textcolor[rgb]{0.67,0.13,1.00}{##1}}}
\expandafter\def\csname PY@tok@s\endcsname{\def\PY@tc##1{\textcolor[rgb]{0.73,0.13,0.13}{##1}}}
\expandafter\def\csname PY@tok@sd\endcsname{\let\PY@it=\textit\def\PY@tc##1{\textcolor[rgb]{0.73,0.13,0.13}{##1}}}
\expandafter\def\csname PY@tok@si\endcsname{\let\PY@bf=\textbf\def\PY@tc##1{\textcolor[rgb]{0.73,0.40,0.53}{##1}}}
\expandafter\def\csname PY@tok@se\endcsname{\let\PY@bf=\textbf\def\PY@tc##1{\textcolor[rgb]{0.73,0.40,0.13}{##1}}}
\expandafter\def\csname PY@tok@sr\endcsname{\def\PY@tc##1{\textcolor[rgb]{0.73,0.40,0.53}{##1}}}
\expandafter\def\csname PY@tok@ss\endcsname{\def\PY@tc##1{\textcolor[rgb]{0.10,0.09,0.49}{##1}}}
\expandafter\def\csname PY@tok@sx\endcsname{\def\PY@tc##1{\textcolor[rgb]{0.00,0.50,0.00}{##1}}}
\expandafter\def\csname PY@tok@m\endcsname{\def\PY@tc##1{\textcolor[rgb]{0.40,0.40,0.40}{##1}}}
\expandafter\def\csname PY@tok@gh\endcsname{\let\PY@bf=\textbf\def\PY@tc##1{\textcolor[rgb]{0.00,0.00,0.50}{##1}}}
\expandafter\def\csname PY@tok@gu\endcsname{\let\PY@bf=\textbf\def\PY@tc##1{\textcolor[rgb]{0.50,0.00,0.50}{##1}}}
\expandafter\def\csname PY@tok@gd\endcsname{\def\PY@tc##1{\textcolor[rgb]{0.63,0.00,0.00}{##1}}}
\expandafter\def\csname PY@tok@gi\endcsname{\def\PY@tc##1{\textcolor[rgb]{0.00,0.63,0.00}{##1}}}
\expandafter\def\csname PY@tok@gr\endcsname{\def\PY@tc##1{\textcolor[rgb]{1.00,0.00,0.00}{##1}}}
\expandafter\def\csname PY@tok@ge\endcsname{\let\PY@it=\textit}
\expandafter\def\csname PY@tok@gs\endcsname{\let\PY@bf=\textbf}
\expandafter\def\csname PY@tok@gp\endcsname{\let\PY@bf=\textbf\def\PY@tc##1{\textcolor[rgb]{0.00,0.00,0.50}{##1}}}
\expandafter\def\csname PY@tok@go\endcsname{\def\PY@tc##1{\textcolor[rgb]{0.53,0.53,0.53}{##1}}}
\expandafter\def\csname PY@tok@gt\endcsname{\def\PY@tc##1{\textcolor[rgb]{0.00,0.27,0.87}{##1}}}
\expandafter\def\csname PY@tok@err\endcsname{\def\PY@bc##1{\setlength{\fboxsep}{0pt}\fcolorbox[rgb]{1.00,0.00,0.00}{1,1,1}{\strut ##1}}}
\expandafter\def\csname PY@tok@kc\endcsname{\let\PY@bf=\textbf\def\PY@tc##1{\textcolor[rgb]{0.00,0.50,0.00}{##1}}}
\expandafter\def\csname PY@tok@kd\endcsname{\let\PY@bf=\textbf\def\PY@tc##1{\textcolor[rgb]{0.00,0.50,0.00}{##1}}}
\expandafter\def\csname PY@tok@kn\endcsname{\let\PY@bf=\textbf\def\PY@tc##1{\textcolor[rgb]{0.00,0.50,0.00}{##1}}}
\expandafter\def\csname PY@tok@kr\endcsname{\let\PY@bf=\textbf\def\PY@tc##1{\textcolor[rgb]{0.00,0.50,0.00}{##1}}}
\expandafter\def\csname PY@tok@bp\endcsname{\def\PY@tc##1{\textcolor[rgb]{0.00,0.50,0.00}{##1}}}
\expandafter\def\csname PY@tok@fm\endcsname{\def\PY@tc##1{\textcolor[rgb]{0.00,0.00,1.00}{##1}}}
\expandafter\def\csname PY@tok@vc\endcsname{\def\PY@tc##1{\textcolor[rgb]{0.10,0.09,0.49}{##1}}}
\expandafter\def\csname PY@tok@vg\endcsname{\def\PY@tc##1{\textcolor[rgb]{0.10,0.09,0.49}{##1}}}
\expandafter\def\csname PY@tok@vi\endcsname{\def\PY@tc##1{\textcolor[rgb]{0.10,0.09,0.49}{##1}}}
\expandafter\def\csname PY@tok@vm\endcsname{\def\PY@tc##1{\textcolor[rgb]{0.10,0.09,0.49}{##1}}}
\expandafter\def\csname PY@tok@sa\endcsname{\def\PY@tc##1{\textcolor[rgb]{0.73,0.13,0.13}{##1}}}
\expandafter\def\csname PY@tok@sb\endcsname{\def\PY@tc##1{\textcolor[rgb]{0.73,0.13,0.13}{##1}}}
\expandafter\def\csname PY@tok@sc\endcsname{\def\PY@tc##1{\textcolor[rgb]{0.73,0.13,0.13}{##1}}}
\expandafter\def\csname PY@tok@dl\endcsname{\def\PY@tc##1{\textcolor[rgb]{0.73,0.13,0.13}{##1}}}
\expandafter\def\csname PY@tok@s2\endcsname{\def\PY@tc##1{\textcolor[rgb]{0.73,0.13,0.13}{##1}}}
\expandafter\def\csname PY@tok@sh\endcsname{\def\PY@tc##1{\textcolor[rgb]{0.73,0.13,0.13}{##1}}}
\expandafter\def\csname PY@tok@s1\endcsname{\def\PY@tc##1{\textcolor[rgb]{0.73,0.13,0.13}{##1}}}
\expandafter\def\csname PY@tok@mb\endcsname{\def\PY@tc##1{\textcolor[rgb]{0.40,0.40,0.40}{##1}}}
\expandafter\def\csname PY@tok@mf\endcsname{\def\PY@tc##1{\textcolor[rgb]{0.40,0.40,0.40}{##1}}}
\expandafter\def\csname PY@tok@mh\endcsname{\def\PY@tc##1{\textcolor[rgb]{0.40,0.40,0.40}{##1}}}
\expandafter\def\csname PY@tok@mi\endcsname{\def\PY@tc##1{\textcolor[rgb]{0.40,0.40,0.40}{##1}}}
\expandafter\def\csname PY@tok@il\endcsname{\def\PY@tc##1{\textcolor[rgb]{0.40,0.40,0.40}{##1}}}
\expandafter\def\csname PY@tok@mo\endcsname{\def\PY@tc##1{\textcolor[rgb]{0.40,0.40,0.40}{##1}}}
\expandafter\def\csname PY@tok@ch\endcsname{\let\PY@it=\textit\def\PY@tc##1{\textcolor[rgb]{0.25,0.50,0.50}{##1}}}
\expandafter\def\csname PY@tok@cm\endcsname{\let\PY@it=\textit\def\PY@tc##1{\textcolor[rgb]{0.25,0.50,0.50}{##1}}}
\expandafter\def\csname PY@tok@cpf\endcsname{\let\PY@it=\textit\def\PY@tc##1{\textcolor[rgb]{0.25,0.50,0.50}{##1}}}
\expandafter\def\csname PY@tok@c1\endcsname{\let\PY@it=\textit\def\PY@tc##1{\textcolor[rgb]{0.25,0.50,0.50}{##1}}}
\expandafter\def\csname PY@tok@cs\endcsname{\let\PY@it=\textit\def\PY@tc##1{\textcolor[rgb]{0.25,0.50,0.50}{##1}}}

\def\PYZbs{\char`\\}
\def\PYZus{\char`\_}
\def\PYZob{\char`\{}
\def\PYZcb{\char`\}}
\def\PYZca{\char`\^}
\def\PYZam{\char`\&}
\def\PYZlt{\char`\<}
\def\PYZgt{\char`\>}
\def\PYZsh{\char`\#}
\def\PYZpc{\char`\%}
\def\PYZdl{\char`\$}
\def\PYZhy{\char`\-}
\def\PYZsq{\char`\'}
\def\PYZdq{\char`\"}
\def\PYZti{\char`\~}
% for compatibility with earlier versions
\def\PYZat{@}
\def\PYZlb{[}
\def\PYZrb{]}
\makeatother


    % Exact colors from NB
    \definecolor{incolor}{rgb}{0.0, 0.0, 0.5}
    \definecolor{outcolor}{rgb}{0.545, 0.0, 0.0}



    
    % Prevent overflowing lines due to hard-to-break entities
    \sloppy 
    % Setup hyperref package
    \hypersetup{
      breaklinks=true,  % so long urls are correctly broken across lines
      colorlinks=true,
      urlcolor=urlcolor,
      linkcolor=linkcolor,
      citecolor=citecolor,
      }
    % Slightly bigger margins than the latex defaults
    
    \geometry{verbose,tmargin=1in,bmargin=1in,lmargin=1in,rmargin=1in}
    
    

    \begin{document}
    
    
    \maketitle
    
    

    
    \section{\texorpdfstring{\textbf{PART 1: PYTHON - A (VERY) SHORT
INTRODUCTION}}{PART 1: PYTHON - A (VERY) SHORT INTRODUCTION}}\label{part-1-python---a-very-short-introduction}

    Today you'll learn the basics of programming in Python. We use a
so-called \emph{Python notebook}. the notebook consists of two types of
cells, those that contain text, like the one you're reading right now,
and those that contain python code.\\
Below is an example of a cell with Python code. When we click on a cell
with code and either press {[}SHIFT{]} + {[}ENTER{]} or click on the
(RUN) button above, we tell the computer to interpret the commands which
means that it turns the Python code into actions we want it to execute.

    \begin{Verbatim}[commandchars=\\\{\}]
{\color{incolor}In [{\color{incolor}1}]:} \PY{c+c1}{\PYZsh{} this is a comment. Everything preceeded by a hash symbol (\PYZsh{}) is ignored by the}
        \PY{c+c1}{\PYZsh{} computer. This allows you to add comments/notes/documentation to your python programme.}
        \PY{n+nb}{print}\PY{p}{(}\PY{l+s+s2}{\PYZdq{}}\PY{l+s+s2}{hello world}\PY{l+s+s2}{\PYZdq{}}\PY{p}{)}
\end{Verbatim}


    \begin{Verbatim}[commandchars=\\\{\}]
hello world

    \end{Verbatim}

    As you can see, we instructed the computer to "print" a string of
letters. Don't worry too much about the syntax right now, later we'll go
through this step by step!

    \subsection{\texorpdfstring{\textbf{1.1
VARIABLES}}{1.1 VARIABLES}}\label{variables}

    Variables are used to store information that can later be referenced and
manipulated by your computer program. It's helpful to think of variables
as containers for information. For example, imagine you build a drawer
and put socks in it. In this case, the drawer would be a variable and
the socks we put in would correspond to its value(s). When you reference
the variable, i.e. open the drawer, you'll find socks in there.

    \begin{Verbatim}[commandchars=\\\{\}]
{\color{incolor}In [{\color{incolor}3}]:} \PY{c+c1}{\PYZsh{} let\PYZsq{}s create a drawer with socks}
        \PY{n}{drawer} \PY{o}{=} \PY{l+s+s2}{\PYZdq{}}\PY{l+s+s2}{socks}\PY{l+s+s2}{\PYZdq{}}
        
        \PY{c+c1}{\PYZsh{} let\PYZsq{}s have a look what\PYZsq{}s inside the drawer}
        \PY{n}{drawer}
\end{Verbatim}


\begin{Verbatim}[commandchars=\\\{\}]
{\color{outcolor}Out[{\color{outcolor}3}]:} 'socks'
\end{Verbatim}
            
    \subsubsection{\texorpdfstring{\textbf{1.1.1 Assign Values to
Variables}}{1.1.1 Assign Values to Variables}}\label{assign-values-to-variables}

    Variables would be useless if you couldn't assign values to them. Python
is able to handle all sorts of different datatypes, such as numbers,
strings (=sequences of letters and symbols) and lists of
numbers/letters. Below are a few examples

    \begin{Verbatim}[commandchars=\\\{\}]
{\color{incolor}In [{\color{incolor}11}]:} \PY{c+c1}{\PYZsh{} this is an integer (a number without any decimal points)}
         \PY{n}{variable1} \PY{o}{=} \PY{l+m+mi}{5} 
         \PY{c+c1}{\PYZsh{} this is another integer}
         \PY{n}{variable2} \PY{o}{=} \PY{l+m+mi}{7}
         \PY{c+c1}{\PYZsh{} this is a float (a number with decimal points)}
         \PY{n}{variable3} \PY{o}{=} \PY{o}{\PYZhy{}}\PY{l+m+mf}{2.563}
         \PY{c+c1}{\PYZsh{} this is a string (a sequences of characters)}
         \PY{n}{variable4} \PY{o}{=} \PY{l+s+s1}{\PYZsq{}}\PY{l+s+s1}{Hello World}\PY{l+s+s1}{\PYZsq{}}
         \PY{c+c1}{\PYZsh{} this is a list (a set of different items)}
         \PY{n}{variable5} \PY{o}{=} \PY{p}{[}\PY{l+m+mi}{1}\PY{p}{,}\PY{l+m+mi}{2}\PY{p}{,} \PY{l+s+s1}{\PYZsq{}}\PY{l+s+s1}{Tomato}\PY{l+s+s1}{\PYZsq{}}\PY{p}{,}\PY{l+m+mi}{0}\PY{p}{,}\PY{l+s+s1}{\PYZsq{}}\PY{l+s+s1}{???}\PY{l+s+s1}{\PYZsq{}}\PY{p}{]}
\end{Verbatim}


    Let's check out what's inside these variables

    \begin{Verbatim}[commandchars=\\\{\}]
{\color{incolor}In [{\color{incolor}7}]:} \PY{n}{variable5}
\end{Verbatim}


\begin{Verbatim}[commandchars=\\\{\}]
{\color{outcolor}Out[{\color{outcolor}7}]:} [1, 2, 'Tomato', 0, '???']
\end{Verbatim}
            
    \textbf{Important!!} Variable names are somewhat arbitrary. But try to
give your variables sensible names that explain what kind of information
they hold (remember the drawer example above). This prevents a lot of
confusion when you come back to code you've written weeks/months ago!

    \subsubsection{\texorpdfstring{\textbf{1.1.2 Operations using defined
variables}}{1.1.2 Operations using defined variables}}\label{operations-using-defined-variables}

    What makes programming languages so powerful is their ability to perform
various operations on the variables you have defined. In this section,
you'll learn a few of those!

    \paragraph{\texorpdfstring{\textbf{Basic addition \&
subtraction}}{Basic addition \& subtraction}}\label{basic-addition-subtraction}

    Let's write a program that adds two of the variables we have defined
above:

    \begin{Verbatim}[commandchars=\\\{\}]
{\color{incolor}In [{\color{incolor}8}]:} \PY{n}{variable1} \PY{o}{+} \PY{n}{variable2} 
\end{Verbatim}


\begin{Verbatim}[commandchars=\\\{\}]
{\color{outcolor}Out[{\color{outcolor}8}]:} 12
\end{Verbatim}
            
    You can conduct basic math directly in-line, but if you need more than
one operation to run, you'll want to assign them to variables:

    \begin{Verbatim}[commandchars=\\\{\}]
{\color{incolor}In [{\color{incolor}9}]:} \PY{n}{sum\PYZus{}of\PYZus{}two} \PY{o}{=} \PY{n}{variable1} \PY{o}{+} \PY{n}{variable2}
        \PY{n}{difference} \PY{o}{=} \PY{n}{variable1} \PY{o}{\PYZhy{}} \PY{n}{variable2}
        \PY{n}{difference2} \PY{o}{=} \PY{n}{difference} \PY{o}{\PYZhy{}} \PY{n}{variable3}
\end{Verbatim}


    Notice how the results of these operations were not automatically
displayed in the command line. This is desired behaviour of any
programming language. Imagine that you have a very long and complicated
script full of mathematical operations, but are only interested in the
end result. It would be a nightmare if each and every intermediate
result was displayed in the command line..

    To display specific variables (and some text) in the command line, use
the \textbf{print} statement.\\
The print statement is a \textbf{function}. As in maths, functions take
some value as input, apply a transformation and return the result of
this transformation as output. In case of the print statement, it takes
strings of characters and variables as input, and - as the name suggests
- "prints" the result to the command line (in our case the space below a
cell with Python code). We'll learn more about functions later in the
course.\\
As you can see in the example below, the print function accepts all
sorts of different input formats. Choose the one that works best for
you!

    \begin{Verbatim}[commandchars=\\\{\}]
{\color{incolor}In [{\color{incolor}12}]:} \PY{c+c1}{\PYZsh{} one variable}
         \PY{n+nb}{print}\PY{p}{(}\PY{n}{difference2}\PY{p}{)} 
         \PY{c+c1}{\PYZsh{} multiple variables can be separated by a comma}
         \PY{n+nb}{print}\PY{p}{(}\PY{n}{sum\PYZus{}of\PYZus{}two}\PY{p}{,} \PY{n}{difference}\PY{p}{)}
         
         \PY{c+c1}{\PYZsh{} we can also construct a string by concatenating text and numbers (that we have turned into strings)}
         \PY{n+nb}{print}\PY{p}{(}\PY{l+s+s1}{\PYZsq{}}\PY{l+s+s1}{Sum of int1 and int2 is }\PY{l+s+s1}{\PYZsq{}} \PY{o}{+} \PY{n+nb}{str}\PY{p}{(}\PY{n}{sum\PYZus{}of\PYZus{}two}\PY{p}{)} \PY{o}{+} \PY{l+s+s1}{\PYZsq{}}\PY{l+s+s1}{ and difference between them is }\PY{l+s+s1}{\PYZsq{}} \PY{o}{+} \PY{n+nb}{str}\PY{p}{(}\PY{n}{difference}\PY{p}{)}\PY{p}{)}
         
         \PY{c+c1}{\PYZsh{} you can also use \PYZob{}\PYZcb{} as placeholders and use the format method to feed values into the placeholders}
         \PY{n+nb}{print}\PY{p}{(}\PY{l+s+s1}{\PYZsq{}}\PY{l+s+s1}{Sum of int1 and int2 is }\PY{l+s+si}{\PYZob{}\PYZcb{}}\PY{l+s+s1}{, and differene between them is }\PY{l+s+si}{\PYZob{}\PYZcb{}}\PY{l+s+s1}{\PYZsq{}}\PY{o}{.}\PY{n}{format}\PY{p}{(}\PY{n}{sum\PYZus{}of\PYZus{}two}\PY{p}{,} \PY{n}{difference}\PY{p}{)}\PY{p}{)}
\end{Verbatim}


    \begin{Verbatim}[commandchars=\\\{\}]
0.5630000000000002
12 -2
Sum of int1 and int2 is 12 and difference between them is -2
Sum of int1 and int2 is 12, and differene between them is -2

    \end{Verbatim}

    \begin{Verbatim}[commandchars=\\\{\}]
{\color{incolor}In [{\color{incolor}14}]:} \PY{c+c1}{\PYZsh{} in case you were confused: Python distinguished between different types of variables. }
         \PY{c+c1}{\PYZsh{} If we assign a number to a variable, Python will assume that it\PYZsq{}s a number.}
         \PY{c+c1}{\PYZsh{} If we put the number into quotation marks, it will assume it\PYZsq{}s a character (= letter on your keyboard) and }
         \PY{c+c1}{\PYZsh{} can\PYZsq{}t do maths with it.}
         \PY{c+c1}{\PYZsh{} \PYZdq{}+\PYZdq{} between numbers adds them up. \PYZdq{}+\PYZdq{} between strings/characters concatenates them}
         \PY{n}{five\PYZus{}num} \PY{o}{=} \PY{l+m+mi}{5}
         \PY{n}{five\PYZus{}str} \PY{o}{=} \PY{l+s+s1}{\PYZsq{}}\PY{l+s+s1}{5}\PY{l+s+s1}{\PYZsq{}} \PY{c+c1}{\PYZsh{} this is equal to str(five\PYZus{}num)}
         \PY{n}{one\PYZus{}num} \PY{o}{=} \PY{l+m+mi}{1}
         \PY{n}{one\PYZus{}str} \PY{o}{=} \PY{l+s+s1}{\PYZsq{}}\PY{l+s+s1}{1}\PY{l+s+s1}{\PYZsq{}}
         \PY{n+nb}{print}\PY{p}{(}\PY{n}{five\PYZus{}num} \PY{o}{+} \PY{n}{one\PYZus{}num}\PY{p}{)}
         \PY{n+nb}{print}\PY{p}{(}\PY{n}{five\PYZus{}str} \PY{o}{+} \PY{n}{one\PYZus{}str}\PY{p}{)}
         \PY{n+nb}{print}\PY{p}{(}\PY{n}{five\PYZus{}num} \PY{o}{+} \PY{n}{five\PYZus{}str}\PY{p}{)}
\end{Verbatim}


    \begin{Verbatim}[commandchars=\\\{\}]
6
51

    \end{Verbatim}

    \begin{Verbatim}[commandchars=\\\{\}]

        ---------------------------------------------------------------------------

        TypeError                                 Traceback (most recent call last)

        <ipython-input-14-d51f6ffb130c> in <module>()
         10 print(five\_num + one\_num)
         11 print(five\_str + one\_str)
    ---> 12 print(five\_num + five\_str)
    

        TypeError: unsupported operand type(s) for +: 'int' and 'str'

    \end{Verbatim}

    \paragraph{\texorpdfstring{\textbf{Basic Multiplication and
Division}}{Basic Multiplication and Division}}\label{basic-multiplication-and-division}

    Imagine how limited a programming language would be, if you couldn't
perform multiplication and division. Luckily, Python has you covered!

    \begin{Verbatim}[commandchars=\\\{\}]
{\color{incolor}In [{\color{incolor}18}]:} \PY{n}{int1} \PY{o}{=} \PY{l+m+mi}{20}
         \PY{n}{int2} \PY{o}{=} \PY{l+m+mi}{6}
         \PY{n}{product\PYZus{}of\PYZus{}two} \PY{o}{=} \PY{n}{int1} \PY{o}{*} \PY{n}{int2}
         \PY{n}{quotient\PYZus{}of\PYZus{}two} \PY{o}{=} \PY{n}{int1} \PY{o}{/} \PY{n}{int2}
         \PY{n}{modulo\PYZus{}of\PYZus{}two} \PY{o}{=} \PY{n}{int1} \PY{o}{\PYZpc{}} \PY{n}{int2}
         \PY{n+nb}{print}\PY{p}{(}\PY{l+s+s1}{\PYZsq{}}\PY{l+s+s1}{Product of int1 and int2 is }\PY{l+s+si}{\PYZob{}\PYZcb{}}\PY{l+s+s1}{\PYZsq{}}\PY{o}{.}\PY{n}{format}\PY{p}{(}\PY{n}{product\PYZus{}of\PYZus{}two}\PY{p}{)}\PY{p}{)}
         \PY{n+nb}{print}\PY{p}{(}\PY{l+s+s1}{\PYZsq{}}\PY{l+s+s1}{Dividing int1 by int2 yields }\PY{l+s+si}{\PYZob{}\PYZcb{}}\PY{l+s+s1}{ }\PY{l+s+s1}{\PYZsq{}}\PY{o}{.}\PY{n}{format}\PY{p}{(}\PY{n}{quotient\PYZus{}of\PYZus{}two}\PY{p}{)}\PY{p}{)}
         \PY{n+nb}{print}\PY{p}{(}\PY{l+s+s1}{\PYZsq{}}\PY{l+s+s1}{The remainder after dividing int1 by int2 is }\PY{l+s+si}{\PYZob{}\PYZcb{}}\PY{l+s+s1}{ }\PY{l+s+s1}{\PYZsq{}}\PY{o}{.}\PY{n}{format}\PY{p}{(}\PY{n}{modulo\PYZus{}of\PYZus{}two}\PY{p}{)}\PY{p}{)}
\end{Verbatim}


    \begin{Verbatim}[commandchars=\\\{\}]
Product of int1 and int2 is 120
Dividing int1 by int2 yields 3.3333333333333335 
The remainder after dividing int1 by int2 is 2 

    \end{Verbatim}

    \paragraph{\texorpdfstring{\textbf{Exponentiation}}{Exponentiation}}\label{exponentiation}

    And of course, you can also perform exponentiation!

    \begin{Verbatim}[commandchars=\\\{\}]
{\color{incolor}In [{\color{incolor}19}]:} \PY{n}{base\PYZus{}1} \PY{o}{=} \PY{l+m+mi}{2}
         \PY{n}{base\PYZus{}2} \PY{o}{=} \PY{l+m+mi}{9}
         \PY{n}{exponent\PYZus{}1} \PY{o}{=} \PY{l+m+mi}{4}
         \PY{n}{exponent\PYZus{}2} \PY{o}{=} \PY{o}{\PYZhy{}}\PY{l+m+mi}{1}
         \PY{n}{exponent\PYZus{}3} \PY{o}{=} \PY{l+m+mf}{0.5}
         \PY{n}{result\PYZus{}1} \PY{o}{=} \PY{n}{base\PYZus{}1}\PY{o}{*}\PY{o}{*}\PY{n}{exponent\PYZus{}1}
         \PY{n}{result\PYZus{}2} \PY{o}{=} \PY{n}{base\PYZus{}1}\PY{o}{*}\PY{o}{*}\PY{n}{exponent\PYZus{}2}
         \PY{n}{result\PYZus{}3} \PY{o}{=} \PY{n}{base\PYZus{}2}\PY{o}{*}\PY{o}{*}\PY{n}{exponent\PYZus{}3}
         \PY{n+nb}{print}\PY{p}{(}\PY{l+s+s1}{\PYZsq{}}\PY{l+s+s1}{2\PYZca{}4= }\PY{l+s+s1}{\PYZsq{}} \PY{o}{+} \PY{n+nb}{str}\PY{p}{(}\PY{n}{result\PYZus{}1}\PY{p}{)}\PY{p}{)}
         \PY{n+nb}{print}\PY{p}{(}\PY{l+s+s1}{\PYZsq{}}\PY{l+s+s1}{2\PYZca{}\PYZhy{}1 = 1/2 = }\PY{l+s+s1}{\PYZsq{}} \PY{o}{+} \PY{n+nb}{str}\PY{p}{(}\PY{n}{result\PYZus{}2}\PY{p}{)}\PY{p}{)}
         \PY{n+nb}{print}\PY{p}{(}\PY{l+s+s1}{\PYZsq{}}\PY{l+s+s1}{9\PYZca{}0.5 = square root of 9 =  }\PY{l+s+s1}{\PYZsq{}} \PY{o}{+} \PY{n+nb}{str}\PY{p}{(}\PY{n}{result\PYZus{}3}\PY{p}{)}\PY{p}{)}
\end{Verbatim}


    \begin{Verbatim}[commandchars=\\\{\}]
2\^{}4= 16
2\^{}-1 = 1/2 = 0.5
9\^{}0.5 = square root of 9 =  3.0

    \end{Verbatim}

    \paragraph{\texorpdfstring{\textbf{Logical
Operators}}{Logical Operators}}\label{logical-operators}

    In programming you very often need to know whether a statement is true
or false. This can be achieved with Boolean variables, which you'll find
become very handy over time:

    \begin{Verbatim}[commandchars=\\\{\}]
{\color{incolor}In [{\color{incolor}20}]:} \PY{n}{bool1} \PY{o}{=} \PY{k+kc}{True}
         \PY{n}{bool2} \PY{o}{=} \PY{k+kc}{False} 
         \PY{n+nb}{print}\PY{p}{(}\PY{n}{bool1}\PY{p}{)}
         \PY{n+nb}{print}\PY{p}{(}\PY{n}{bool2}\PY{p}{)}
         \PY{n+nb}{print}\PY{p}{(}\PY{n}{bool1} \PY{o+ow}{and} \PY{n}{bool2}\PY{p}{)}
         \PY{n+nb}{print}\PY{p}{(}\PY{n}{bool1} \PY{o+ow}{or} \PY{n}{bool2}\PY{p}{)}
         \PY{n+nb}{print}\PY{p}{(}\PY{n}{bool1} \PY{o+ow}{and} \PY{o+ow}{not} \PY{n}{bool2}\PY{p}{)}
\end{Verbatim}


    \begin{Verbatim}[commandchars=\\\{\}]
True
False
False
True
True

    \end{Verbatim}

    More frequently, you will be able to check whether relationships between
variables are true or false or do something conditionally based on
whether a statement is true (see below):

    \begin{Verbatim}[commandchars=\\\{\}]
{\color{incolor}In [{\color{incolor}21}]:} \PY{n+nb}{print}\PY{p}{(}\PY{l+m+mi}{5} \PY{o}{\PYZlt{}} \PY{l+m+mi}{6}\PY{p}{)}
         \PY{n+nb}{print}\PY{p}{(}\PY{l+m+mi}{6} \PY{o}{\PYZgt{}}\PY{o}{=} \PY{l+m+mi}{5}\PY{p}{)}
         \PY{n+nb}{print}\PY{p}{(}\PY{l+m+mi}{5} \PY{o}{!=} \PY{l+m+mi}{10}\PY{o}{/}\PY{l+m+mi}{2}\PY{p}{)}
         \PY{n+nb}{print}\PY{p}{(}\PY{l+s+s1}{\PYZsq{}}\PY{l+s+s1}{hello}\PY{l+s+s1}{\PYZsq{}} \PY{o}{!=} \PY{l+s+s1}{\PYZsq{}}\PY{l+s+s1}{world}\PY{l+s+s1}{\PYZsq{}}\PY{p}{)}
\end{Verbatim}


    \begin{Verbatim}[commandchars=\\\{\}]
True
True
False
True

    \end{Verbatim}

    Logical operators are very powerful. For instance, you can test whether
a variable lies within a certain interval:

    \begin{Verbatim}[commandchars=\\\{\}]
{\color{incolor}In [{\color{incolor}22}]:} \PY{n}{int1} \PY{o}{=} \PY{l+m+mi}{10}
         \PY{n+nb}{print}\PY{p}{(}\PY{n}{int1} \PY{o}{\PYZlt{}} \PY{l+m+mi}{12} \PY{o+ow}{and} \PY{n}{int1} \PY{o}{\PYZgt{}} \PY{l+m+mi}{9}\PY{p}{)}
         \PY{n+nb}{print}\PY{p}{(} \PY{l+m+mi}{9} \PY{o}{\PYZlt{}} \PY{n}{int1} \PY{o}{\PYZlt{}} \PY{l+m+mi}{12}\PY{p}{)}
\end{Verbatim}


    \begin{Verbatim}[commandchars=\\\{\}]
True
True

    \end{Verbatim}

    \subsubsection{\texorpdfstring{\textbf{TASK: Define variables and write
code
to...}}{TASK: Define variables and write code to...}}\label{task-define-variables-and-write-code-to...}

\begin{enumerate}
\def\labelenumi{\arabic{enumi}.}
\tightlist
\item
  Calculate square root of 25, 144 and 289
\item
  Translate the following into Python
\end{enumerate}

\begin{itemize}
\tightlist
\item
  assign numbers between 1 and 100 to 3 different variables
\item
  add up your 3 variables and print whether the sum is bigger than 100
\end{itemize}

\textbf{Hints}: * You can either use a built-in function (sqrt(x)), or
perform clever exponentiation (what does x\^{}0.5 do to x?) * look at
the section on logical operators if you're stuck!

    \begin{Verbatim}[commandchars=\\\{\}]
{\color{incolor}In [{\color{incolor}2}]:} \PY{c+c1}{\PYZsh{} Your code for example 1 here }
        \PY{n+nb}{print}\PY{p}{(}\PY{l+m+mi}{25}\PY{o}{*}\PY{o}{*}\PY{l+m+mf}{0.5}\PY{p}{)}
        \PY{n+nb}{print}\PY{p}{(}\PY{l+m+mi}{144}\PY{o}{*}\PY{o}{*}\PY{o}{.}\PY{l+m+mi}{5}\PY{p}{)}
        \PY{n+nb}{print}\PY{p}{(}\PY{l+m+mi}{289}\PY{o}{*}\PY{o}{*}\PY{o}{.}\PY{l+m+mi}{5}\PY{p}{)}
\end{Verbatim}


    \begin{Verbatim}[commandchars=\\\{\}]
5.0
12.0
17.0

    \end{Verbatim}

    \begin{Verbatim}[commandchars=\\\{\}]
{\color{incolor}In [{\color{incolor}3}]:} \PY{c+c1}{\PYZsh{} Your code for example 2 here }
        \PY{n}{var1} \PY{o}{=} \PY{l+m+mi}{20}
        \PY{n}{var2} \PY{o}{=} \PY{l+m+mi}{66}
        \PY{n}{var3} \PY{o}{=} \PY{l+m+mi}{2}
        \PY{n+nb}{print}\PY{p}{(}\PY{p}{(}\PY{n}{var1} \PY{o}{+} \PY{n}{var2} \PY{o}{+} \PY{n}{var3}\PY{p}{)} \PY{o}{\PYZgt{}} \PY{l+m+mi}{100}\PY{p}{)}
\end{Verbatim}


    \begin{Verbatim}[commandchars=\\\{\}]
False

    \end{Verbatim}

    \subsection{\texorpdfstring{\textbf{1.2 LOOPS and
STATEMENTS}}{1.2 LOOPS and STATEMENTS}}\label{loops-and-statements}

    Loops allow you to execute the same operation multiple times, but with a
different type of input on each iteration.\\
Statement allow you to structure your code, and have the computer care
about certain bits only if conditions you specify are met.\\
In this section, we'll learn more about both

    \subsubsection{\texorpdfstring{\textbf{1.2.1 While
Loops}}{1.2.1 While Loops}}\label{while-loops}

    Use while loops whenever you need to apply an operation multiple times,
but don't know exactly how many iterations are needed:

    \begin{Verbatim}[commandchars=\\\{\}]
{\color{incolor}In [{\color{incolor}4}]:} \PY{c+c1}{\PYZsh{} WHILE loops}
        \PY{n}{x} \PY{o}{=} \PY{l+m+mi}{500}
        \PY{k}{while} \PY{n}{x} \PY{o}{\PYZgt{}} \PY{l+m+mi}{10}\PY{p}{:}
            \PY{n+nb}{print}\PY{p}{(}\PY{n+nb}{str}\PY{p}{(}\PY{n}{x}\PY{p}{)}\PY{p}{)}
            \PY{n}{x} \PY{o}{=} \PY{n}{x} \PY{o}{/}\PY{o}{/} \PY{l+m+mi}{2}
\end{Verbatim}


    \begin{Verbatim}[commandchars=\\\{\}]
500
250
125
62
31
15

    \end{Verbatim}

    \subsubsection{\texorpdfstring{\textbf{1.2.2 For
Loops}}{1.2.2 For Loops}}\label{for-loops}

    For loops come in handy whenever you need to apply a function to each
item of a (potentially very long) list, and you know exactly how many
iterations you need: (note for the pros: all for loops in python are
actually for-each loops)

    \begin{Verbatim}[commandchars=\\\{\}]
{\color{incolor}In [{\color{incolor}26}]:} \PY{c+c1}{\PYZsh{} FOR loops }
         
         \PY{n}{modules} \PY{o}{=} \PY{p}{[}\PY{l+s+s2}{\PYZdq{}}\PY{l+s+s2}{cognitive}\PY{l+s+s2}{\PYZdq{}}\PY{p}{,} \PY{l+s+s2}{\PYZdq{}}\PY{l+s+s2}{developmental}\PY{l+s+s2}{\PYZdq{}}\PY{p}{,} \PY{l+s+s2}{\PYZdq{}}\PY{l+s+s2}{neuro}\PY{l+s+s2}{\PYZdq{}}\PY{p}{,} \PY{l+s+s2}{\PYZdq{}}\PY{l+s+s2}{clinical}\PY{l+s+s2}{\PYZdq{}}\PY{p}{,} \PY{l+s+s2}{\PYZdq{}}\PY{l+s+s2}{social}\PY{l+s+s2}{\PYZdq{}}\PY{p}{,} \PY{l+s+s2}{\PYZdq{}}\PY{l+s+s2}{computational}\PY{l+s+s2}{\PYZdq{}}\PY{p}{]}
         \PY{k}{for} \PY{n}{x} \PY{o+ow}{in} \PY{n}{modules}\PY{p}{:}
             \PY{n+nb}{print}\PY{p}{(}\PY{n}{x}\PY{p}{)}
\end{Verbatim}


    \begin{Verbatim}[commandchars=\\\{\}]
cognitive
developmental
neuro
clinical
social
computational

    \end{Verbatim}

    \subsubsection{\texorpdfstring{\textbf{1.2.3 If...Then
Conditionals}}{1.2.3 If...Then Conditionals}}\label{if...then-conditionals}

    Before we talk about if..then conditionals, let's first learn something
new! the len() command counts the number if symbols in a list. Example:

    \begin{Verbatim}[commandchars=\\\{\}]
{\color{incolor}In [{\color{incolor}27}]:} \PY{c+c1}{\PYZsh{} let\PYZsq{}s count how many numbers are in this list}
         \PY{n+nb}{print}\PY{p}{(}\PY{n+nb}{len}\PY{p}{(}\PY{p}{[}\PY{l+m+mi}{4}\PY{p}{,}\PY{l+m+mi}{5}\PY{p}{,}\PY{l+m+mi}{5}\PY{p}{,}\PY{l+m+mi}{2}\PY{p}{,}\PY{l+m+mi}{1}\PY{p}{]}\PY{p}{)}\PY{p}{)}
         \PY{c+c1}{\PYZsh{} Note that it also works with strings (which are essentially lists of symbols)}
         \PY{n+nb}{print}\PY{p}{(}\PY{n+nb}{len}\PY{p}{(}\PY{l+s+s1}{\PYZsq{}}\PY{l+s+s1}{hello world}\PY{l+s+s1}{\PYZsq{}}\PY{p}{)}\PY{p}{)}
\end{Verbatim}


    \begin{Verbatim}[commandchars=\\\{\}]
5
11

    \end{Verbatim}

    Programming is all about case-specific processing. Use if..then
statements to process inputs depending on conditions you have specified:

    \begin{Verbatim}[commandchars=\\\{\}]
{\color{incolor}In [{\color{incolor}28}]:} \PY{c+c1}{\PYZsh{} IF...THEN statements }
         
         
         \PY{n}{favoriteModule} \PY{o}{=} \PY{l+s+s2}{\PYZdq{}}\PY{l+s+s2}{computational}\PY{l+s+s2}{\PYZdq{}}
         \PY{n}{firstModule} \PY{o}{=} \PY{l+s+s2}{\PYZdq{}}\PY{l+s+s2}{social}\PY{l+s+s2}{\PYZdq{}}
         
         \PY{k}{if} \PY{n+nb}{len}\PY{p}{(}\PY{n}{firstModule}\PY{p}{)} \PY{o}{\PYZlt{}} \PY{n+nb}{len}\PY{p}{(}\PY{n}{favoriteModule}\PY{p}{)}\PY{p}{:} 
             \PY{n+nb}{print}\PY{p}{(}\PY{l+s+s1}{\PYZsq{}}\PY{l+s+s1}{These modules are of different lengths!}\PY{l+s+s1}{\PYZsq{}}\PY{p}{)}
         \PY{k}{elif} \PY{n+nb}{len}\PY{p}{(}\PY{n}{firstModule}\PY{p}{)} \PY{o}{==} \PY{n+nb}{len}\PY{p}{(}\PY{n}{favoriteModule}\PY{p}{)}\PY{p}{:} 
             \PY{n+nb}{print}\PY{p}{(}\PY{l+s+s1}{\PYZsq{}}\PY{l+s+s1}{These modules are of same lengths!}\PY{l+s+s1}{\PYZsq{}}\PY{p}{)}
\end{Verbatim}


    \begin{Verbatim}[commandchars=\\\{\}]
These modules are of different lengths!

    \end{Verbatim}

    Very commonly in coding practice you'll find that you want to check
multiple conditions before doing something with your variables. Python
is easy when it comes to this - just tell it whether you want to use
'and' or 'or' or a different logical connector.

\href{https://https://docs.python.org/2.0/ref/lambda.html}{Link to a
good starting point on Boolean operators in Python}

    \begin{Verbatim}[commandchars=\\\{\}]
{\color{incolor}In [{\color{incolor}29}]:} \PY{n}{a} \PY{o}{=} \PY{l+m+mi}{200}
         \PY{n}{b} \PY{o}{=} \PY{l+m+mi}{33}
         \PY{n}{c} \PY{o}{=} \PY{l+m+mi}{500}
         \PY{k}{if} \PY{n}{a} \PY{o}{\PYZgt{}} \PY{n}{b} \PY{o+ow}{and} \PY{n}{c} \PY{o}{\PYZgt{}} \PY{n}{a}\PY{p}{:}
             \PY{n+nb}{print}\PY{p}{(}\PY{l+s+s1}{\PYZsq{}}\PY{l+s+s1}{Both conditions are true!}\PY{l+s+s1}{\PYZsq{}}\PY{p}{)}
         \PY{k}{else}\PY{p}{:} 
             \PY{n+nb}{print}\PY{p}{(}\PY{l+s+s1}{\PYZsq{}}\PY{l+s+s1}{Both conditions are not true!}\PY{l+s+s1}{\PYZsq{}}\PY{p}{)}
           
           
         \PY{k}{if} \PY{n}{a} \PY{o}{\PYZgt{}} \PY{n}{b} \PY{o+ow}{or} \PY{n}{a} \PY{o}{\PYZgt{}} \PY{n}{c}\PY{p}{:}
             \PY{n+nb}{print}\PY{p}{(}\PY{l+s+s1}{\PYZsq{}}\PY{l+s+s1}{At least one of the conditions is true!}\PY{l+s+s1}{\PYZsq{}}\PY{p}{)}
\end{Verbatim}


    \begin{Verbatim}[commandchars=\\\{\}]
Both conditions are true!
At least one of the conditions is true!

    \end{Verbatim}

    \subsubsection{\texorpdfstring{\textbf{TASK: Write your own for loop and
if/then
statement}}{TASK: Write your own for loop and if/then statement}}\label{task-write-your-own-for-loop-and-ifthen-statement}

Given

\begin{verbatim}
 modules = ["cognitive", "developmental", "neuro", "clinical", "social", "computational"]
 comparisonVariable = "introductory"
\end{verbatim}

Determine whether any of the modules have names that are longer than the
comparison variable.

    \begin{Verbatim}[commandchars=\\\{\}]
{\color{incolor}In [{\color{incolor}6}]:} \PY{n}{modules} \PY{o}{=} \PY{p}{[}\PY{l+s+s2}{\PYZdq{}}\PY{l+s+s2}{cognitive}\PY{l+s+s2}{\PYZdq{}}\PY{p}{,} \PY{l+s+s2}{\PYZdq{}}\PY{l+s+s2}{developmental}\PY{l+s+s2}{\PYZdq{}}\PY{p}{,} \PY{l+s+s2}{\PYZdq{}}\PY{l+s+s2}{neuro}\PY{l+s+s2}{\PYZdq{}}\PY{p}{,} \PY{l+s+s2}{\PYZdq{}}\PY{l+s+s2}{clinical}\PY{l+s+s2}{\PYZdq{}}\PY{p}{,} \PY{l+s+s2}{\PYZdq{}}\PY{l+s+s2}{social}\PY{l+s+s2}{\PYZdq{}}\PY{p}{,} \PY{l+s+s2}{\PYZdq{}}\PY{l+s+s2}{computational}\PY{l+s+s2}{\PYZdq{}}\PY{p}{]}
        \PY{n}{comparisonVariable} \PY{o}{=} \PY{l+s+s2}{\PYZdq{}}\PY{l+s+s2}{introductory}\PY{l+s+s2}{\PYZdq{}}
        
        \PY{c+c1}{\PYZsh{} Your code here}
        \PY{n}{anyNameLonger} \PY{o}{=} \PY{k+kc}{False}
        \PY{k}{for} \PY{n}{mod} \PY{o+ow}{in} \PY{n}{modules}\PY{p}{:}
            \PY{k}{if} \PY{n+nb}{len}\PY{p}{(}\PY{n}{mod}\PY{p}{)} \PY{o}{\PYZgt{}} \PY{n+nb}{len}\PY{p}{(}\PY{n}{comparisonVariable}\PY{p}{)}\PY{p}{:}
                \PY{n}{anyNameLonger} \PY{o}{=} \PY{k+kc}{True}
        \PY{n+nb}{print}\PY{p}{(}\PY{n}{anyNameLonger}\PY{p}{)}
\end{Verbatim}


    \begin{Verbatim}[commandchars=\\\{\}]
True

    \end{Verbatim}

    \subsection{\texorpdfstring{\textbf{1.3
FUNCTIONS}}{1.3 FUNCTIONS}}\label{functions}

    \subsubsection{\texorpdfstring{\textbf{1.3.1 Introduction to
Functions}}{1.3.1 Introduction to Functions}}\label{introduction-to-functions}

    Functions are nothing new. You have already worked with them in the
previous exercises!

Here are some examples...

    \begin{Verbatim}[commandchars=\\\{\}]
{\color{incolor}In [{\color{incolor}31}]:} \PY{n+nb}{print}\PY{p}{(}\PY{l+s+s2}{\PYZdq{}}\PY{l+s+s2}{hello world}\PY{l+s+s2}{\PYZdq{}}\PY{p}{)}
         \PY{n+nb}{print}\PY{p}{(}\PY{l+s+s2}{\PYZdq{}}\PY{l+s+s2}{2*3:}\PY{l+s+s2}{\PYZdq{}}\PY{p}{,}\PY{l+m+mi}{2}\PY{o}{*}\PY{l+m+mi}{3}\PY{p}{)}
         \PY{n+nb}{print}\PY{p}{(}\PY{n+nb}{len}\PY{p}{(}\PY{l+s+s2}{\PYZdq{}}\PY{l+s+s2}{hello world}\PY{l+s+s2}{\PYZdq{}}\PY{p}{)}\PY{p}{)}
\end{Verbatim}


    \begin{Verbatim}[commandchars=\\\{\}]
hello world
2*3: 6
11

    \end{Verbatim}

    What makes coding so great, is that you're able to write your own
functions! Whenever you identify a recurring task in your workflow,
consider writing a function instead of copying \& pasting code over and
over again:

    \begin{Verbatim}[commandchars=\\\{\}]
{\color{incolor}In [{\color{incolor}32}]:} \PY{c+c1}{\PYZsh{} naive approach for testing parity of several numbers:}
         \PY{k}{if} \PY{l+m+mi}{3} \PY{o}{\PYZpc{}} \PY{l+m+mi}{2} \PY{o}{==} \PY{l+m+mi}{0}\PY{p}{:}
             \PY{n+nb}{print}\PY{p}{(}\PY{l+s+s1}{\PYZsq{}}\PY{l+s+si}{\PYZob{}\PYZcb{}}\PY{l+s+s1}{ is even}\PY{l+s+s1}{\PYZsq{}}\PY{o}{.}\PY{n}{format}\PY{p}{(}\PY{l+m+mi}{3}\PY{p}{)}\PY{p}{)}
         \PY{k}{else}\PY{p}{:}
             \PY{n+nb}{print}\PY{p}{(}\PY{l+s+s1}{\PYZsq{}}\PY{l+s+si}{\PYZob{}\PYZcb{}}\PY{l+s+s1}{ is odd}\PY{l+s+s1}{\PYZsq{}}\PY{o}{.}\PY{n}{format}\PY{p}{(}\PY{l+m+mi}{3}\PY{p}{)}\PY{p}{)}
         
             
         \PY{k}{if} \PY{l+m+mi}{2} \PY{o}{\PYZpc{}} \PY{l+m+mi}{2} \PY{o}{==} \PY{l+m+mi}{0}\PY{p}{:}
             \PY{n+nb}{print}\PY{p}{(}\PY{l+s+s1}{\PYZsq{}}\PY{l+s+si}{\PYZob{}\PYZcb{}}\PY{l+s+s1}{ is even}\PY{l+s+s1}{\PYZsq{}}\PY{o}{.}\PY{n}{format}\PY{p}{(}\PY{l+m+mi}{2}\PY{p}{)}\PY{p}{)}
         \PY{k}{else}\PY{p}{:}
             \PY{n+nb}{print}\PY{p}{(}\PY{l+s+s1}{\PYZsq{}}\PY{l+s+si}{\PYZob{}\PYZcb{}}\PY{l+s+s1}{ is odd}\PY{l+s+s1}{\PYZsq{}}\PY{o}{.}\PY{n}{format}\PY{p}{(}\PY{l+m+mi}{2}\PY{p}{)}\PY{p}{)}
         
         
         \PY{k}{if} \PY{l+m+mi}{11} \PY{o}{\PYZpc{}} \PY{l+m+mi}{2} \PY{o}{==} \PY{l+m+mi}{0}\PY{p}{:}
             \PY{n+nb}{print}\PY{p}{(}\PY{l+s+s1}{\PYZsq{}}\PY{l+s+si}{\PYZob{}\PYZcb{}}\PY{l+s+s1}{ is even}\PY{l+s+s1}{\PYZsq{}}\PY{o}{.}\PY{n}{format}\PY{p}{(}\PY{l+m+mi}{11}\PY{p}{)}\PY{p}{)}
         \PY{k}{else}\PY{p}{:}
             \PY{n+nb}{print}\PY{p}{(}\PY{l+s+s1}{\PYZsq{}}\PY{l+s+si}{\PYZob{}\PYZcb{}}\PY{l+s+s1}{ is odd}\PY{l+s+s1}{\PYZsq{}}\PY{o}{.}\PY{n}{format}\PY{p}{(}\PY{l+m+mi}{11}\PY{p}{)}\PY{p}{)}
\end{Verbatim}


    \begin{Verbatim}[commandchars=\\\{\}]
3 is odd
2 is even
11 is odd

    \end{Verbatim}

    In the example above, we tested whether several numbers are odd or even.
Instead of repeating the same if..then construct over and over again, we
can put it in a neat little function:

    \begin{Verbatim}[commandchars=\\\{\}]
{\color{incolor}In [{\color{incolor}33}]:} \PY{c+c1}{\PYZsh{} let\PYZsq{}s write a function that does the work for us!}
         \PY{k}{def} \PY{n+nf}{return\PYZus{}parity}\PY{p}{(}\PY{n}{n}\PY{p}{)}\PY{p}{:}
             \PY{n}{result} \PY{o}{=} \PY{n+nb}{str}\PY{p}{(}\PY{n}{n}\PY{p}{)} \PY{o}{+} \PY{l+s+s1}{\PYZsq{}}\PY{l+s+s1}{ is}\PY{l+s+s1}{\PYZsq{}}
             \PY{k}{if} \PY{n}{n} \PY{o}{\PYZpc{}} \PY{l+m+mi}{2} \PY{o}{==} \PY{l+m+mi}{0}\PY{p}{:}
                 \PY{k}{return} \PY{n}{result} \PY{o}{+} \PY{l+s+s1}{\PYZsq{}}\PY{l+s+s1}{ even}\PY{l+s+s1}{\PYZsq{}}
             \PY{k}{else}\PY{p}{:}
                 \PY{k}{return} \PY{n}{result} \PY{o}{+} \PY{l+s+s1}{\PYZsq{}}\PY{l+s+s1}{ odd}\PY{l+s+s1}{\PYZsq{}}
           
\end{Verbatim}


    Now we can simply apply the function to each of the numbers we're
interested in

    \begin{Verbatim}[commandchars=\\\{\}]
{\color{incolor}In [{\color{incolor}34}]:} \PY{n+nb}{print}\PY{p}{(}\PY{n}{return\PYZus{}parity}\PY{p}{(}\PY{l+m+mi}{3}\PY{p}{)}\PY{p}{)}
         \PY{n+nb}{print}\PY{p}{(}\PY{n}{return\PYZus{}parity}\PY{p}{(}\PY{l+m+mi}{2}\PY{p}{)}\PY{p}{)}
         \PY{n+nb}{print}\PY{p}{(}\PY{n}{return\PYZus{}parity}\PY{p}{(}\PY{l+m+mi}{11}\PY{p}{)}\PY{p}{)}
\end{Verbatim}


    \begin{Verbatim}[commandchars=\\\{\}]
3 is odd
2 is even
11 is odd

    \end{Verbatim}

    By the way, if..then statements can be compressed into one-liners:

    \begin{Verbatim}[commandchars=\\\{\}]
{\color{incolor}In [{\color{incolor}35}]:} \PY{c+c1}{\PYZsh{} let\PYZsq{}s make this shorter}
         \PY{k}{def} \PY{n+nf}{return\PYZus{}parity}\PY{p}{(}\PY{n}{n}\PY{p}{)}\PY{p}{:}
             \PY{n}{result} \PY{o}{=} \PY{n+nb}{str}\PY{p}{(}\PY{n}{n}\PY{p}{)} \PY{o}{+} \PY{l+s+s1}{\PYZsq{}}\PY{l+s+s1}{ is }\PY{l+s+s1}{\PYZsq{}}
             \PY{k}{return} \PY{n}{result} \PY{o}{+} \PY{l+s+s1}{\PYZsq{}}\PY{l+s+s1}{even}\PY{l+s+s1}{\PYZsq{}} \PY{k}{if} \PY{n}{n} \PY{o}{\PYZpc{}} \PY{l+m+mi}{2} \PY{o}{==} \PY{l+m+mi}{0} \PY{k}{else} \PY{n}{result} \PY{o}{+} \PY{l+s+s1}{\PYZsq{}}\PY{l+s+s1}{odd}\PY{l+s+s1}{\PYZsq{}}
\end{Verbatim}


    Did you notice something? We applied the function to several different
numbers. Why not put all numbers in a list and use a loop to iterate
over them?

    \begin{Verbatim}[commandchars=\\\{\}]
{\color{incolor}In [{\color{incolor}36}]:} \PY{c+c1}{\PYZsh{} let\PYZsq{}s combine this with loops}
         \PY{n}{numbers} \PY{o}{=} \PY{p}{[}\PY{l+m+mi}{3}\PY{p}{,} \PY{l+m+mi}{2}\PY{p}{,} \PY{l+m+mi}{11}\PY{p}{]}
         \PY{k}{for} \PY{n}{ii} \PY{o+ow}{in} \PY{n}{numbers}\PY{p}{:}
             \PY{n+nb}{print}\PY{p}{(}\PY{n}{return\PYZus{}parity}\PY{p}{(}\PY{n}{ii}\PY{p}{)}\PY{p}{)}
\end{Verbatim}


    \begin{Verbatim}[commandchars=\\\{\}]
3 is odd
2 is even
11 is odd

    \end{Verbatim}

    Functions can be put inside other functions. Let's write a wrapper that
iterates over a list!

    \begin{Verbatim}[commandchars=\\\{\}]
{\color{incolor}In [{\color{incolor}37}]:} \PY{c+c1}{\PYZsh{}let\PYZsq{}s write a wrapper!}
         \PY{k}{def} \PY{n+nf}{check\PYZus{}item\PYZus{}parity}\PY{p}{(}\PY{n}{arr}\PY{p}{)}\PY{p}{:}
             \PY{k}{for} \PY{n}{ii} \PY{o+ow}{in} \PY{n}{arr}\PY{p}{:}
                 \PY{n+nb}{print}\PY{p}{(}\PY{n}{return\PYZus{}parity}\PY{p}{(}\PY{n}{ii}\PY{p}{)}\PY{p}{)}
           
\end{Verbatim}


    \begin{Verbatim}[commandchars=\\\{\}]
{\color{incolor}In [{\color{incolor}38}]:} \PY{n}{check\PYZus{}item\PYZus{}parity}\PY{p}{(}\PY{n}{numbers}\PY{p}{)}
\end{Verbatim}


    \begin{Verbatim}[commandchars=\\\{\}]
3 is odd
2 is even
11 is odd

    \end{Verbatim}

    Now if you define a different list of numbers you want to check parity
for, you can simply use your wrapped function instead of calling
check\_parity over and over again. See below:

    \begin{Verbatim}[commandchars=\\\{\}]
{\color{incolor}In [{\color{incolor}39}]:} \PY{n}{longerArray} \PY{o}{=} \PY{p}{[}\PY{l+m+mi}{2}\PY{p}{,} \PY{l+m+mi}{3}\PY{p}{,} \PY{l+m+mi}{4}\PY{p}{,} \PY{l+m+mi}{5}\PY{p}{,} \PY{l+m+mi}{6}\PY{p}{,} \PY{l+m+mi}{7}\PY{p}{,} \PY{l+m+mi}{8}\PY{p}{,} \PY{l+m+mi}{9}\PY{p}{,} \PY{l+m+mi}{10}\PY{p}{,} \PY{l+m+mi}{23}\PY{p}{,} \PY{l+m+mi}{25}\PY{p}{,} \PY{l+m+mi}{72}\PY{p}{,} \PY{l+m+mi}{43}\PY{p}{,} \PY{l+m+mi}{52}\PY{p}{,} \PY{l+m+mi}{26}\PY{p}{,} \PY{l+m+mi}{24}\PY{p}{,} \PY{l+m+mi}{23}\PY{p}{,} \PY{l+m+mi}{16}\PY{p}{]}
         \PY{n}{check\PYZus{}item\PYZus{}parity}\PY{p}{(}\PY{n}{longerArray}\PY{p}{)}
\end{Verbatim}


    \begin{Verbatim}[commandchars=\\\{\}]
2 is even
3 is odd
4 is even
5 is odd
6 is even
7 is odd
8 is even
9 is odd
10 is even
23 is odd
25 is odd
72 is even
43 is odd
52 is even
26 is even
24 is even
23 is odd
16 is even

    \end{Verbatim}

    Functions are treated as so-called first class objects - you can assign
them to variables...

    \begin{Verbatim}[commandchars=\\\{\}]
{\color{incolor}In [{\color{incolor}40}]:} \PY{n}{my\PYZus{}len} \PY{o}{=} \PY{n+nb}{len}
         \PY{n+nb}{print}\PY{p}{(}\PY{n}{my\PYZus{}len}\PY{p}{(}\PY{l+s+s2}{\PYZdq{}}\PY{l+s+s2}{hello world}\PY{l+s+s2}{\PYZdq{}}\PY{p}{)}\PY{p}{)}
         
         \PY{c+c1}{\PYZsh{} (don\PYZsq{}t ask why you would want to do this. but it works x) )}
\end{Verbatim}


    \begin{Verbatim}[commandchars=\\\{\}]
11

    \end{Verbatim}

    .. and you can pass them as inputs to other functions

    \begin{Verbatim}[commandchars=\\\{\}]
{\color{incolor}In [{\color{incolor}41}]:} \PY{k}{def} \PY{n+nf}{do\PYZus{}sth\PYZus{}with\PYZus{}number}\PY{p}{(}\PY{n}{number}\PY{p}{,} \PY{n}{something}\PY{p}{)}\PY{p}{:}
             \PY{n+nb}{print}\PY{p}{(}\PY{n}{something}\PY{p}{(}\PY{n}{number}\PY{p}{)}\PY{p}{)}
\end{Verbatim}


    \begin{Verbatim}[commandchars=\\\{\}]
{\color{incolor}In [{\color{incolor}42}]:} \PY{n}{do\PYZus{}sth\PYZus{}with\PYZus{}number}\PY{p}{(}\PY{o}{\PYZhy{}}\PY{l+m+mi}{2}\PY{p}{,} \PY{n+nb}{abs}\PY{p}{)}
         \PY{n}{do\PYZus{}sth\PYZus{}with\PYZus{}number}\PY{p}{(}\PY{l+m+mf}{9.34}\PY{p}{,}\PY{n+nb}{round}\PY{p}{)}
         \PY{n}{do\PYZus{}sth\PYZus{}with\PYZus{}number}\PY{p}{(}\PY{l+m+mi}{9}\PY{p}{,}\PY{n}{return\PYZus{}parity}\PY{p}{)}
\end{Verbatim}


    \begin{Verbatim}[commandchars=\\\{\}]
2
9
9 is odd

    \end{Verbatim}

    \subsubsection{\texorpdfstring{\textbf{1.3.2
Docstrings}}{1.3.2 Docstrings}}\label{docstrings}

    Programming can be a very complicated endeavour. And who is able to
remember for each and every function how to use it? Nobody. In fact, in
real life, whenever you need to write a program, you'll notice that you
spend most of the time googling stuff and scrolling through online
forums. That's absolutely normal and even the biggest coding pros do
this all the time.\\
But even if you don't have internet, there's still hope: Python provides
documentation for all of its built-in functions. These brief info
snippets are called \emph{docstrings} and we'll show you two ways to
access them.\\
First, you could simply add a question mark to the function you're
interested in, and execute the line of code:

    \begin{Verbatim}[commandchars=\\\{\}]
{\color{incolor}In [{\color{incolor}43}]:} print\PY{o}{?}
\end{Verbatim}


    Docstring: print(value, ..., sep=' ', end='\n', file=sys.stdout,
flush=False)

Prints the values to a stream, or to sys.stdout by default. Optional
keyword arguments: file: a file-like object (stream); defaults to the
current sys.stdout. sep: string inserted between values, default a
space. end: string appended after the last value, default a newline.
flush: whether to forcibly flush the stream. Type:
builtin\_function\_or\_method

    Alternatively, you pass it as argument to the help function:

    \begin{Verbatim}[commandchars=\\\{\}]
{\color{incolor}In [{\color{incolor}44}]:} \PY{n}{help}\PY{p}{(}\PY{n+nb}{print}\PY{p}{)}
\end{Verbatim}


    \begin{Verbatim}[commandchars=\\\{\}]
Help on built-in function print in module builtins:

print({\ldots})
    print(value, {\ldots}, sep=' ', end='\textbackslash{}n', file=sys.stdout, flush=False)
    
    Prints the values to a stream, or to sys.stdout by default.
    Optional keyword arguments:
    file:  a file-like object (stream); defaults to the current sys.stdout.
    sep:   string inserted between values, default a space.
    end:   string appended after the last value, default a newline.
    flush: whether to forcibly flush the stream.


    \end{Verbatim}

    When you write your own functions, don't forget to add documentation on
how to use them. This saves a lot of time and avoids nervous breakdowns
when you come back to your code a few months later. In fact, you can
easily write your own docstrings! Here's an example\\
Docstrings are indicated by """ yourtextgoeshere """

    \begin{Verbatim}[commandchars=\\\{\}]
{\color{incolor}In [{\color{incolor}45}]:} \PY{k}{def} \PY{n+nf}{return\PYZus{}parity}\PY{p}{(}\PY{n}{n}\PY{p}{)}\PY{p}{:}
             \PY{l+s+sd}{\PYZdq{}\PYZdq{}\PYZdq{} }
         \PY{l+s+sd}{      This function tests whether its input is odd or even.}
         \PY{l+s+sd}{      INPUT: (int) an arbitrary number}
         \PY{l+s+sd}{      OUTPUT: (str) \PYZsq{}even\PYZsq{} or \PYZsq{}odd\PYZsq{} }
         \PY{l+s+sd}{    \PYZdq{}\PYZdq{}\PYZdq{}}
             \PY{n}{result} \PY{o}{=} \PY{n+nb}{str}\PY{p}{(}\PY{n}{n}\PY{p}{)} \PY{o}{+} \PY{l+s+s1}{\PYZsq{}}\PY{l+s+s1}{ is }\PY{l+s+s1}{\PYZsq{}}
             \PY{k}{return} \PY{n}{result} \PY{o}{+} \PY{l+s+s1}{\PYZsq{}}\PY{l+s+s1}{even}\PY{l+s+s1}{\PYZsq{}} \PY{k}{if} \PY{n}{n} \PY{o}{\PYZpc{}} \PY{l+m+mi}{2} \PY{o}{==} \PY{l+m+mi}{0} \PY{k}{else} \PY{n}{result} \PY{o}{+} \PY{l+s+s1}{\PYZsq{}}\PY{l+s+s1}{odd}\PY{l+s+s1}{\PYZsq{}}
\end{Verbatim}


    \begin{Verbatim}[commandchars=\\\{\}]
{\color{incolor}In [{\color{incolor}46}]:} \PY{n}{help}\PY{p}{(}\PY{n}{return\PYZus{}parity}\PY{p}{)}
\end{Verbatim}


    \begin{Verbatim}[commandchars=\\\{\}]
Help on function return\_parity in module \_\_main\_\_:

return\_parity(n)
    This function tests whether its input is odd or even.
    INPUT: (int) an arbitrary number
    OUTPUT: (str) 'even' or 'odd'


    \end{Verbatim}

    \subsubsection{\texorpdfstring{\textbf{Task: Write your own
function!}}{Task: Write your own function!}}\label{task-write-your-own-function}

    Write a function that takes a list of numbers as inputs and returns the
mean of these numbers!\\
Remember, given a vector of numbers \textgreater{}
\(\textbf{x} = [x_1, x_2, x_3, x_4, ...., x_n]\)

the mean is defined as:

\begin{quote}
\(mean(\textbf{x})=\frac{1}{N}\sum_{i=1}^{N} x_i = \frac{1}{N}(x_1 + x_2 + x_3 + ... + x_n)\)
\end{quote}

    \begin{Verbatim}[commandchars=\\\{\}]
{\color{incolor}In [{\color{incolor}7}]:} \PY{n}{numbers} \PY{o}{=} \PY{p}{[}\PY{l+m+mi}{2}\PY{p}{,}\PY{l+m+mi}{5}\PY{p}{,}\PY{l+m+mi}{8}\PY{p}{,}\PY{l+m+mi}{7}\PY{p}{,}\PY{l+m+mi}{1}\PY{p}{,}\PY{l+m+mi}{4}\PY{p}{,}\PY{l+m+mi}{4}\PY{p}{,}\PY{l+m+mi}{9}\PY{p}{]}
        \PY{c+c1}{\PYZsh{} your code goes here}
        \PY{c+c1}{\PYZsh{} SAMPLE SOLUTION}
        \PY{k}{def} \PY{n+nf}{compute\PYZus{}mean}\PY{p}{(}\PY{n}{x}\PY{p}{)}\PY{p}{:}
            \PY{k}{return} \PY{n+nb}{sum}\PY{p}{(}\PY{n}{x}\PY{p}{)}\PY{o}{/}\PY{n+nb}{len}\PY{p}{(}\PY{n}{x}\PY{p}{)}
\end{Verbatim}


    Now write a function that computes the variance of a list of numbers.
Remember, the variance is defined as \textgreater{}
\(var(\textbf{x}) = \frac{1}{N} \sum_{i=1}^{N}(x_i-\bar{x})^2\), where
\(mean(\textbf{x}) = \bar{x}\)\\
\textgreater{} \textgreater{} or alternatively:\\
\textgreater{} \textgreater{}
\(var(\textbf{x}) = \frac{1}{N} \sum_{i=1}^{N}(x_i^2) -\bar{x}^2\)
("expectation of square minus square of expectation")

    \begin{Verbatim}[commandchars=\\\{\}]
{\color{incolor}In [{\color{incolor}8}]:} \PY{n}{numbers} \PY{o}{=} \PY{p}{[}\PY{l+m+mi}{2}\PY{p}{,}\PY{l+m+mi}{5}\PY{p}{,}\PY{l+m+mi}{8}\PY{p}{,}\PY{l+m+mi}{7}\PY{p}{,}\PY{l+m+mi}{1}\PY{p}{,}\PY{l+m+mi}{4}\PY{p}{,}\PY{l+m+mi}{4}\PY{p}{,}\PY{l+m+mi}{9}\PY{p}{]}
        \PY{c+c1}{\PYZsh{} your code goes here}
        \PY{c+c1}{\PYZsh{} SAMPLE SOLUTION}
        \PY{k}{def} \PY{n+nf}{compute\PYZus{}variance}\PY{p}{(}\PY{n}{x}\PY{p}{)}\PY{p}{:}  
            \PY{k}{return} \PY{n}{compute\PYZus{}mean}\PY{p}{(}\PY{p}{[}\PY{n}{ii}\PY{o}{*}\PY{o}{*}\PY{l+m+mi}{2} \PY{k}{for} \PY{n}{ii} \PY{o+ow}{in} \PY{n}{x}\PY{p}{]}\PY{p}{)}\PY{o}{\PYZhy{}}\PY{n}{compute\PYZus{}mean}\PY{p}{(}\PY{n}{x}\PY{p}{)}\PY{o}{*}\PY{o}{*}\PY{l+m+mi}{2}
            \PY{n+nb}{print}\PY{p}{(}\PY{n}{compute\PYZus{}variance}\PY{p}{(}\PY{n}{numbers}\PY{p}{)}\PY{p}{)}
\end{Verbatim}


    Now add docstrings to your functions to explain how to use them

    \begin{Verbatim}[commandchars=\\\{\}]
{\color{incolor}In [{\color{incolor}9}]:} \PY{c+c1}{\PYZsh{}SAMPLE SOLUTIONS}
        \PY{k}{def} \PY{n+nf}{compute\PYZus{}mean}\PY{p}{(}\PY{n}{x}\PY{p}{)}\PY{p}{:}
            \PY{l+s+sd}{\PYZsq{}\PYZsq{}\PYZsq{} }
        \PY{l+s+sd}{    computes mean of a list of numbers }
        \PY{l+s+sd}{    INPUT: x (list)}
        \PY{l+s+sd}{    OUTPUT: mu (float)}
        \PY{l+s+sd}{    \PYZsq{}\PYZsq{}\PYZsq{}}
            \PY{k}{return} \PY{n+nb}{sum}\PY{p}{(}\PY{n}{x}\PY{p}{)}\PY{o}{/}\PY{n+nb}{len}\PY{p}{(}\PY{n}{x}\PY{p}{)}
        
        \PY{k}{def} \PY{n+nf}{compute\PYZus{}variance}\PY{p}{(}\PY{n}{x}\PY{p}{)}\PY{p}{:}  
            \PY{l+s+sd}{\PYZsq{}\PYZsq{}\PYZsq{}}
        \PY{l+s+sd}{    computes variance of a list of numbers }
        \PY{l+s+sd}{    INPUT: x (list)}
        \PY{l+s+sd}{    OUTPUT: var (float)}
        \PY{l+s+sd}{    \PYZsq{}\PYZsq{}\PYZsq{}}
            \PY{k}{return} \PY{n}{compute\PYZus{}mean}\PY{p}{(}\PY{p}{[}\PY{n}{ii}\PY{o}{*}\PY{o}{*}\PY{l+m+mi}{2} \PY{k}{for} \PY{n}{ii} \PY{o+ow}{in} \PY{n}{x}\PY{p}{]}\PY{p}{)}\PY{o}{\PYZhy{}}\PY{n}{compute\PYZus{}mean}\PY{p}{(}\PY{n}{x}\PY{p}{)}\PY{o}{*}\PY{o}{*}\PY{l+m+mi}{2}
        
        \PY{n+nb}{print}\PY{p}{(}\PY{n}{help}\PY{p}{(}\PY{n}{compute\PYZus{}mean}\PY{p}{)}\PY{p}{)}
        \PY{n+nb}{print}\PY{p}{(}\PY{n}{help}\PY{p}{(}\PY{n}{compute\PYZus{}variance}\PY{p}{)}\PY{p}{)}
\end{Verbatim}


    \begin{Verbatim}[commandchars=\\\{\}]
Help on function compute\_mean in module \_\_main\_\_:

compute\_mean(x)
    computes mean of a list of numbers 
    INPUT: x (list)
    OUTPUT: mu (float)

None
Help on function compute\_variance in module \_\_main\_\_:

compute\_variance(x)
    computes variance of a list of numbers 
    INPUT: x (list)
    OUTPUT: var (float)

None

    \end{Verbatim}

    \subsubsection{\texorpdfstring{\textbf{BONUS: Advanced Programming
Principle:
Recursion}}{BONUS: Advanced Programming Principle: Recursion}}\label{bonus-advanced-programming-principle-recursion}

    Functions are very versatile! They can even call themselves with a
subset of of the original problem. This is called recursion. Below is a
naive solution to find the factorial of a number. As reminder, a
factorial of a number n is just a product of all integers from 1 to n:\\
**n! = 1\emph{2}3\emph{4}......*n**

    \begin{Verbatim}[commandchars=\\\{\}]
{\color{incolor}In [{\color{incolor}49}]:} \PY{k}{def} \PY{n+nf}{iterative\PYZus{}factorial}\PY{p}{(}\PY{n}{n}\PY{p}{)}\PY{p}{:}
             \PY{n}{fact} \PY{o}{=} \PY{l+m+mi}{1}
             \PY{k}{for} \PY{n}{ii} \PY{o+ow}{in} \PY{n+nb}{range}\PY{p}{(}\PY{n}{n}\PY{p}{)}\PY{p}{:}
                 \PY{n}{fact} \PY{o}{*}\PY{o}{=} \PY{n}{ii}\PY{o}{+}\PY{l+m+mi}{1}
             \PY{k}{return} \PY{n}{fact}
\end{Verbatim}


    \begin{Verbatim}[commandchars=\\\{\}]
{\color{incolor}In [{\color{incolor}50}]:} \PY{n+nb}{print}\PY{p}{(}\PY{l+s+s2}{\PYZdq{}}\PY{l+s+s2}{4!: }\PY{l+s+s2}{\PYZdq{}}\PY{p}{,}\PY{n}{iterative\PYZus{}factorial}\PY{p}{(}\PY{l+m+mi}{4}\PY{p}{)}\PY{p}{)}
\end{Verbatim}


    \begin{Verbatim}[commandchars=\\\{\}]
4!:  24

    \end{Verbatim}

    \begin{Verbatim}[commandchars=\\\{\}]
{\color{incolor}In [{\color{incolor}51}]:} \PY{k}{def} \PY{n+nf}{recursive\PYZus{}factorial}\PY{p}{(}\PY{n}{n}\PY{p}{)}\PY{p}{:}
             \PY{k}{if} \PY{n}{n} \PY{o}{\PYZgt{}}\PY{l+m+mi}{1}\PY{p}{:}
                 \PY{n}{fact} \PY{o}{=} \PY{n}{n}\PY{o}{*}\PY{n}{recursive\PYZus{}factorial}\PY{p}{(}\PY{n}{n}\PY{o}{\PYZhy{}}\PY{l+m+mi}{1}\PY{p}{)}
                 \PY{k}{return} \PY{n}{fact}
             \PY{k}{else}\PY{p}{:}
                 \PY{k}{return} \PY{n}{n}
             
\end{Verbatim}


    \begin{Verbatim}[commandchars=\\\{\}]
{\color{incolor}In [{\color{incolor}52}]:} \PY{n+nb}{print}\PY{p}{(}\PY{l+s+s2}{\PYZdq{}}\PY{l+s+s2}{4!: }\PY{l+s+s2}{\PYZdq{}}\PY{p}{,}\PY{n}{recursive\PYZus{}factorial}\PY{p}{(}\PY{l+m+mi}{4}\PY{p}{)}\PY{p}{)}
\end{Verbatim}


    \begin{Verbatim}[commandchars=\\\{\}]
4!:  24

    \end{Verbatim}

    \subsection{\texorpdfstring{\textbf{1.4
.METHODS()}}{1.4 .METHODS()}}\label{methods}

    Python is an object oriented language. An object is an abstract entity
that contains some data and functions that can be applied to the data
(Imagine a drawer (container) for socks (data) that has a built-in sock
sorting machine (function)). Functions that are applied directly to
objects are called \textbf{methods}. In this section we'll learn how to
apply methods to objects.\\
Many entities in Python are objects. For example, strings are objects as
they contain some data, as for example "hello world", and they come with
many functions that can be applied to the data. Let's have a look at an
example:

    \subsubsection{\texorpdfstring{\textbf{1.4.1 Example: methods to
manipulate
strings}}{1.4.1 Example: methods to manipulate strings}}\label{example-methods-to-manipulate-strings}

    \begin{Verbatim}[commandchars=\\\{\}]
{\color{incolor}In [{\color{incolor}53}]:} \PY{c+c1}{\PYZsh{} here we create a simple variable that contains a sequence of symbols}
         \PY{n}{string} \PY{o}{=} \PY{l+s+s2}{\PYZdq{}}\PY{l+s+s2}{hello world}\PY{l+s+s2}{\PYZdq{}}
         \PY{c+c1}{\PYZsh{} Python understands that this is what it would call a string, and allows you to apply methods}
         \PY{c+c1}{\PYZsh{} that are purpose\PYZhy{}built for string manipulation:}
         \PY{n+nb}{print}\PY{p}{(}\PY{n}{string}\PY{o}{.}\PY{n}{upper}\PY{p}{(}\PY{p}{)}\PY{p}{)}
         \PY{n+nb}{print}\PY{p}{(}\PY{n}{string}\PY{o}{.}\PY{n}{capitalize}\PY{p}{(}\PY{p}{)}\PY{p}{)}
         \PY{n+nb}{print}\PY{p}{(}\PY{n}{string}\PY{o}{.}\PY{n}{replace}\PY{p}{(}\PY{l+s+s1}{\PYZsq{}}\PY{l+s+s1}{ world}\PY{l+s+s1}{\PYZsq{}}\PY{p}{,}\PY{l+s+s1}{\PYZsq{}}\PY{l+s+s1}{, nice to meet you}\PY{l+s+s1}{\PYZsq{}}\PY{p}{)}\PY{p}{)}
         \PY{n}{listOfWords} \PY{o}{=} \PY{n}{string}\PY{o}{.}\PY{n}{split}\PY{p}{(}\PY{l+s+s1}{\PYZsq{}}\PY{l+s+s1}{ }\PY{l+s+s1}{\PYZsq{}}\PY{p}{)}
         \PY{n+nb}{print}\PY{p}{(}\PY{n}{listOfWords}\PY{p}{)}
         \PY{n}{newString} \PY{o}{=} \PY{l+s+s1}{\PYZsq{}}\PY{l+s+s1}{, }\PY{l+s+s1}{\PYZsq{}}\PY{o}{.}\PY{n}{join}\PY{p}{(}\PY{n}{listOfWords}\PY{p}{)}
         \PY{n+nb}{print}\PY{p}{(}\PY{n}{newString}\PY{p}{)}
\end{Verbatim}


    \begin{Verbatim}[commandchars=\\\{\}]
HELLO WORLD
Hello world
hello, nice to meet you
['hello', 'world']
hello, world

    \end{Verbatim}

    in contrast to functions, for which the syntax is
\emph{function(argument)}, methods are bound to objects. Thus, the
syntax is \emph{object.method(argument)}

    \subsubsection{\texorpdfstring{\textbf{1.4.2 Methods are object
specific!}}{1.4.2 Methods are object specific!}}\label{methods-are-object-specific}

    As methods are bound to a particular object type, you can't apply them
to other objects. For instance, you can't capitalise numbers:

    \begin{Verbatim}[commandchars=\\\{\}]
{\color{incolor}In [{\color{incolor}54}]:} \PY{c+c1}{\PYZsh{} let\PYZsq{}s create a variable that contains a number}
         \PY{n}{number} \PY{o}{=} \PY{l+m+mf}{0.5}
         \PY{c+c1}{\PYZsh{} Python realises that this is not a string, and therefore you\PYZsq{}re not able to capitalise it!}
         \PY{n}{number}\PY{o}{.}\PY{n}{capitalize}\PY{p}{(}\PY{p}{)}
\end{Verbatim}


    \begin{Verbatim}[commandchars=\\\{\}]

        ---------------------------------------------------------------------------

        AttributeError                            Traceback (most recent call last)

        <ipython-input-54-93d524e9ed5f> in <module>()
          2 number = 0.5
          3 \# Python realises that this is not a string, and therefore you're not able to capitalise it!
    ----> 4 number.capitalize()
    

        AttributeError: 'float' object has no attribute 'capitalize'

    \end{Verbatim}

    \subsubsection{\texorpdfstring{\textbf{1.4.3 How to figure out which
methods are provided for a given
object}}{1.4.3 How to figure out which methods are provided for a given object}}\label{how-to-figure-out-which-methods-are-provided-for-a-given-object}

    to see a list of all available methods, either press {[}TAB{]} after
placing a dot behind an object, or use the following syntax

    \begin{Verbatim}[commandchars=\\\{\}]
{\color{incolor}In [{\color{incolor}77}]:} \PY{n+nb}{dir}\PY{p}{(}\PY{n+nb}{str}\PY{p}{)}
\end{Verbatim}


\begin{Verbatim}[commandchars=\\\{\}]
{\color{outcolor}Out[{\color{outcolor}77}]:} ['\_\_add\_\_',
          '\_\_class\_\_',
          '\_\_contains\_\_',
          '\_\_delattr\_\_',
          '\_\_dir\_\_',
          '\_\_doc\_\_',
          '\_\_eq\_\_',
          '\_\_format\_\_',
          '\_\_ge\_\_',
          '\_\_getattribute\_\_',
          '\_\_getitem\_\_',
          '\_\_getnewargs\_\_',
          '\_\_gt\_\_',
          '\_\_hash\_\_',
          '\_\_init\_\_',
          '\_\_init\_subclass\_\_',
          '\_\_iter\_\_',
          '\_\_le\_\_',
          '\_\_len\_\_',
          '\_\_lt\_\_',
          '\_\_mod\_\_',
          '\_\_mul\_\_',
          '\_\_ne\_\_',
          '\_\_new\_\_',
          '\_\_reduce\_\_',
          '\_\_reduce\_ex\_\_',
          '\_\_repr\_\_',
          '\_\_rmod\_\_',
          '\_\_rmul\_\_',
          '\_\_setattr\_\_',
          '\_\_sizeof\_\_',
          '\_\_str\_\_',
          '\_\_subclasshook\_\_',
          'capitalize',
          'casefold',
          'center',
          'count',
          'encode',
          'endswith',
          'expandtabs',
          'find',
          'format',
          'format\_map',
          'index',
          'isalnum',
          'isalpha',
          'isdecimal',
          'isdigit',
          'isidentifier',
          'islower',
          'isnumeric',
          'isprintable',
          'isspace',
          'istitle',
          'isupper',
          'join',
          'ljust',
          'lower',
          'lstrip',
          'maketrans',
          'partition',
          'replace',
          'rfind',
          'rindex',
          'rjust',
          'rpartition',
          'rsplit',
          'rstrip',
          'split',
          'splitlines',
          'startswith',
          'strip',
          'swapcase',
          'title',
          'translate',
          'upper',
          'zfill']
\end{Verbatim}
            
    \begin{Verbatim}[commandchars=\\\{\}]
{\color{incolor}In [{\color{incolor}78}]:} \PY{n}{help}\PY{p}{(}\PY{n+nb}{str}\PY{o}{.}\PY{n}{split}\PY{p}{)}
\end{Verbatim}


    \begin{Verbatim}[commandchars=\\\{\}]
Help on method\_descriptor:

split({\ldots})
    S.split(sep=None, maxsplit=-1) -> list of strings
    
    Return a list of the words in S, using sep as the
    delimiter string.  If maxsplit is given, at most maxsplit
    splits are done. If sep is not specified or is None, any
    whitespace string is a separator and empty strings are
    removed from the result.


    \end{Verbatim}

    \subsubsection{\texorpdfstring{\textbf{Task: String
Manipulation}}{Task: String Manipulation}}\label{task-string-manipulation}

You're given the following two strings:

\begin{Shaded}
\begin{Highlighting}[]
\NormalTok{gibberish_1 }\OperatorTok{=} \StringTok{'is much I can''t enough.'}
\NormalTok{gibberish_2 }\OperatorTok{=} \StringTok{'cOdinG sO FuN. JuST gEt '}
\end{Highlighting}
\end{Shaded}

\begin{itemize}
\tightlist
\item
  turn the strings into lists of words (use the .split method)
\item
  make sure that gibberish\_2 contains only lower case letters (.lower()
  method)
\item
  create a new empty list ( assign {[}{]} to a variable)
\item
  loop through 5 steps and for each iteration, add pairs of words to
  your new empty list
\item
  turn the list into a string (use the ' '.join(list) method)
\item
  print the result
\end{itemize}

    \begin{Verbatim}[commandchars=\\\{\}]
{\color{incolor}In [{\color{incolor}55}]:} \PY{n}{gibberish\PYZus{}1} \PY{o}{=} \PY{l+s+s1}{\PYZsq{}}\PY{l+s+s1}{is much I can}\PY{l+s+s1}{\PYZsq{}}\PY{l+s+s1}{\PYZsq{}}\PY{l+s+s1}{t enough.}\PY{l+s+s1}{\PYZsq{}}
         \PY{n}{gibberish\PYZus{}2} \PY{o}{=} \PY{l+s+s1}{\PYZsq{}}\PY{l+s+s1}{cOdinG sO FuN. JuST gEt }\PY{l+s+s1}{\PYZsq{}}
         
         \PY{c+c1}{\PYZsh{} Your code goes here}
         \PY{c+c1}{\PYZsh{} SAMPLE SOLUTION 1}
         \PY{n}{seq1} \PY{o}{=} \PY{n}{gibberish\PYZus{}1}\PY{o}{.}\PY{n}{split}\PY{p}{(}\PY{p}{)} 
         \PY{n}{seq2} \PY{o}{=} \PY{n}{gibberish\PYZus{}2}\PY{o}{.}\PY{n}{lower}\PY{p}{(}\PY{p}{)}\PY{o}{.}\PY{n}{split}\PY{p}{(}\PY{p}{)}
         \PY{n}{wordlist} \PY{o}{=} \PY{p}{[}\PY{p}{]}
         \PY{k}{for} \PY{n}{ii} \PY{o+ow}{in} \PY{n+nb}{range}\PY{p}{(}\PY{n+nb}{len}\PY{p}{(}\PY{n}{seq1}\PY{p}{)}\PY{p}{)}\PY{p}{:}
             \PY{n}{wordlist}\PY{o}{.}\PY{n}{append}\PY{p}{(}\PY{l+s+s1}{\PYZsq{}}\PY{l+s+s1}{ }\PY{l+s+s1}{\PYZsq{}}\PY{o}{.}\PY{n}{join}\PY{p}{(}\PY{p}{(}\PY{n}{seq2}\PY{p}{[}\PY{n}{ii}\PY{p}{]}\PY{p}{,} \PY{n}{seq1}\PY{p}{[}\PY{n}{ii}\PY{p}{]}\PY{p}{)}\PY{p}{)}\PY{p}{)}
         \PY{n}{sentence} \PY{o}{=} \PY{l+s+s1}{\PYZsq{}}\PY{l+s+s1}{ }\PY{l+s+s1}{\PYZsq{}}\PY{o}{.}\PY{n}{join}\PY{p}{(}\PY{n}{wordlist}\PY{p}{)}
         \PY{n+nb}{print}\PY{p}{(}\PY{n}{sentence}\PY{p}{)}
         
         \PY{c+c1}{\PYZsh{} SAMPLE SOLUTION 2}
         \PY{n}{seq1} \PY{o}{=} \PY{n}{gibberish\PYZus{}1}\PY{o}{.}\PY{n}{split}\PY{p}{(}\PY{p}{)} 
         \PY{n}{seq2} \PY{o}{=} \PY{n}{gibberish\PYZus{}2}\PY{o}{.}\PY{n}{lower}\PY{p}{(}\PY{p}{)}\PY{o}{.}\PY{n}{split}\PY{p}{(}\PY{p}{)}
         \PY{n}{wordlist} \PY{o}{=} \PY{p}{[}\PY{p}{]}
         \PY{k}{for} \PY{n}{ii} \PY{o+ow}{in} \PY{n+nb}{range}\PY{p}{(}\PY{n+nb}{len}\PY{p}{(}\PY{n}{seq1}\PY{p}{)}\PY{p}{)}\PY{p}{:}
             \PY{n}{wordlist}\PY{o}{.}\PY{n}{append}\PY{p}{(}\PY{n}{seq2}\PY{p}{[}\PY{n}{ii}\PY{p}{]}\PY{p}{)}
             \PY{n}{wordlist}\PY{o}{.}\PY{n}{append}\PY{p}{(}\PY{n}{seq1}\PY{p}{[}\PY{n}{ii}\PY{p}{]}\PY{p}{)}
         \PY{n}{sentence} \PY{o}{=} \PY{l+s+s1}{\PYZsq{}}\PY{l+s+s1}{ }\PY{l+s+s1}{\PYZsq{}}\PY{o}{.}\PY{n}{join}\PY{p}{(}\PY{n}{wordlist}\PY{p}{)}
         \PY{n+nb}{print}\PY{p}{(}\PY{n}{sentence}\PY{p}{)}
         
         \PY{c+c1}{\PYZsh{} SAMPLE SOLUTION 3 }
         \PY{n}{seq1} \PY{o}{=} \PY{n}{gibberish\PYZus{}1}\PY{o}{.}\PY{n}{split}\PY{p}{(}\PY{p}{)} 
         \PY{n}{seq2} \PY{o}{=} \PY{n}{gibberish\PYZus{}2}\PY{o}{.}\PY{n}{lower}\PY{p}{(}\PY{p}{)}\PY{o}{.}\PY{n}{split}\PY{p}{(}\PY{p}{)}
         \PY{n+nb}{print}\PY{p}{(}\PY{l+s+s1}{\PYZsq{}}\PY{l+s+s1}{ }\PY{l+s+s1}{\PYZsq{}}\PY{o}{.}\PY{n}{join}\PY{p}{(}\PY{p}{[}\PY{l+s+s1}{\PYZsq{}}\PY{l+s+s1}{ }\PY{l+s+s1}{\PYZsq{}}\PY{o}{.}\PY{n}{join}\PY{p}{(}\PY{p}{(}\PY{n}{seq2}\PY{p}{[}\PY{n}{ii}\PY{p}{]}\PY{p}{,}\PY{n}{seq1}\PY{p}{[}\PY{n}{ii}\PY{p}{]}\PY{p}{)}\PY{p}{)} \PY{k}{for} \PY{n}{ii} \PY{o+ow}{in} \PY{n+nb}{range}\PY{p}{(}\PY{n+nb}{len}\PY{p}{(}\PY{n}{seq1}\PY{p}{)}\PY{p}{)}\PY{p}{]}\PY{p}{)}\PY{p}{)}
\end{Verbatim}


    \begin{Verbatim}[commandchars=\\\{\}]
coding is so much fun. I just cant get enough.
coding is so much fun. I just cant get enough.
coding is so much fun. I just cant get enough.

    \end{Verbatim}

    \section{\texorpdfstring{\textbf{PART2: PYTHON FOR SCIENTISTS:
PACKAGES/LIBRARIES}}{PART2: PYTHON FOR SCIENTISTS: PACKAGES/LIBRARIES}}\label{part2-python-for-scientists-packageslibraries}

    Some of the most powerful tools you'll use in Python are packages and
libraries. These contain useful functions - from quite simple to more
advanced ones. We'll go through a few basic examples today, but leave
you with a list of libraries and packages to check out:

\begin{itemize}
\tightlist
\item
  numpy
\item
  matplotlib
\item
  seaborn
\item
  scipy
\end{itemize}

    \begin{Verbatim}[commandchars=\\\{\}]
{\color{incolor}In [{\color{incolor}56}]:} \PY{c+c1}{\PYZsh{}You\PYZsq{}ll need to import every package you want to use. }
         \PY{c+c1}{\PYZsh{} This is not a one\PYZhy{}time thing \PYZhy{} you have to do it in each new Python notebook. }
         \PY{c+c1}{\PYZsh{} With time you\PYZsq{}ll have a pretty good sense of what packages you need }
         \PY{c+c1}{\PYZsh{}(and Stackoverflow to the rescue, always!). }
         \PY{k+kn}{import} \PY{n+nn}{numpy}
\end{Verbatim}


    \begin{Verbatim}[commandchars=\\\{\}]
{\color{incolor}In [{\color{incolor}57}]:} \PY{n}{array} \PY{o}{=} \PY{p}{[}\PY{l+m+mi}{2}\PY{p}{,} \PY{l+m+mi}{4}\PY{p}{,} \PY{l+m+mi}{6}\PY{p}{,} \PY{l+m+mi}{8}\PY{p}{,} \PY{l+m+mi}{10}\PY{p}{]}
         
         \PY{n}{averageOfArray} \PY{o}{=} \PY{n}{numpy}\PY{o}{.}\PY{n}{mean}\PY{p}{(}\PY{n}{array}\PY{p}{)}
         \PY{n+nb}{print}\PY{p}{(}\PY{n}{averageOfArray}\PY{p}{)}
\end{Verbatim}


    \begin{Verbatim}[commandchars=\\\{\}]
6.0

    \end{Verbatim}

    \begin{Verbatim}[commandchars=\\\{\}]
{\color{incolor}In [{\color{incolor}59}]:} \PY{c+c1}{\PYZsh{} Alternatively, you can use abbreviations \PYZhy{} this is really handy. }
         \PY{k+kn}{import} \PY{n+nn}{numpy} \PY{k}{as} \PY{n+nn}{np}
         
         \PY{n}{averageOfArray} \PY{o}{=} \PY{n}{np}\PY{o}{.}\PY{n}{mean}\PY{p}{(}\PY{n}{array}\PY{p}{)}
         \PY{n+nb}{print}\PY{p}{(}\PY{n}{averageOfArray}\PY{p}{)}
\end{Verbatim}


    \begin{Verbatim}[commandchars=\\\{\}]
6.0

    \end{Verbatim}

    \begin{Verbatim}[commandchars=\\\{\}]
{\color{incolor}In [{\color{incolor}60}]:} \PY{c+c1}{\PYZsh{} And, of course, you can print or use any of these directly, such as: }
         \PY{n+nb}{print}\PY{p}{(}\PY{n}{np}\PY{o}{.}\PY{n}{mean}\PY{p}{(}\PY{n}{array}\PY{p}{)}\PY{p}{)}
\end{Verbatim}


    \begin{Verbatim}[commandchars=\\\{\}]
6.0

    \end{Verbatim}

    \begin{Verbatim}[commandchars=\\\{\}]
{\color{incolor}In [{\color{incolor}61}]:} \PY{c+c1}{\PYZsh{} Numpy is also powerful in prepopulating your arrays}
         \PY{n}{a} \PY{o}{=} \PY{n}{np}\PY{o}{.}\PY{n}{zeros}\PY{p}{(}\PY{p}{(}\PY{l+m+mi}{2}\PY{p}{)}\PY{p}{)}  
         \PY{n+nb}{print}\PY{p}{(}\PY{n}{a}\PY{p}{)} 
         
         \PY{n+nb}{print}\PY{p}{(}\PY{l+s+s1}{\PYZsq{}}\PY{l+s+s1}{\PYZsq{}}\PY{p}{)} \PY{c+c1}{\PYZsh{} space to separate outputs}
         
         \PY{c+c1}{\PYZsh{} or even something like if you need multiple dimensions}
         \PY{n}{b} \PY{o}{=} \PY{n}{np}\PY{o}{.}\PY{n}{ones}\PY{p}{(}\PY{p}{(}\PY{l+m+mi}{2}\PY{p}{,} \PY{l+m+mi}{2}\PY{p}{)}\PY{p}{)}
         \PY{n+nb}{print}\PY{p}{(}\PY{n}{b}\PY{p}{)}
         
         \PY{n+nb}{print}\PY{p}{(}\PY{l+s+s1}{\PYZsq{}}\PY{l+s+s1}{\PYZsq{}}\PY{p}{)}
         
         \PY{c+c1}{\PYZsh{} you can also create an array filled with the same number of choice, something like }
         \PY{n}{x} \PY{o}{=} \PY{l+m+mf}{27.34} 
         \PY{n}{c} \PY{o}{=} \PY{n}{np}\PY{o}{.}\PY{n}{full}\PY{p}{(}\PY{p}{(}\PY{l+m+mi}{5}\PY{p}{)}\PY{p}{,} \PY{n}{x}\PY{p}{)}
         \PY{n+nb}{print}\PY{p}{(}\PY{n}{c}\PY{p}{)}
\end{Verbatim}


    \begin{Verbatim}[commandchars=\\\{\}]
[0. 0.]

[[1. 1.]
 [1. 1.]]

[27.34 27.34 27.34 27.34 27.34]

    \end{Verbatim}

    A few ways to visualize your data and important things to note

    \begin{Verbatim}[commandchars=\\\{\}]
{\color{incolor}In [{\color{incolor}62}]:} \PY{c+c1}{\PYZsh{} Let\PYZsq{}s start by plotting something quite simple using a line plot in matplotlib (remember to import this in every notebook)}
         \PY{k+kn}{import} \PY{n+nn}{matplotlib}\PY{n+nn}{.}\PY{n+nn}{pyplot} \PY{k}{as} \PY{n+nn}{plt}
         \PY{n}{x} \PY{o}{=} \PY{p}{[}\PY{l+m+mi}{1}\PY{p}{,} \PY{l+m+mi}{2}\PY{p}{,} \PY{l+m+mi}{3}\PY{p}{,} \PY{l+m+mi}{4}\PY{p}{,} \PY{l+m+mi}{5}\PY{p}{]}
         \PY{n}{y} \PY{o}{=} \PY{p}{[}\PY{l+m+mi}{2}\PY{p}{,} \PY{l+m+mi}{5}\PY{p}{,} \PY{l+m+mi}{6}\PY{p}{,} \PY{l+m+mi}{11}\PY{p}{,} \PY{l+m+mi}{14}\PY{p}{]}
         
         \PY{n}{plt}\PY{o}{.}\PY{n}{plot}\PY{p}{(}\PY{n}{x}\PY{p}{,} \PY{n}{y}\PY{p}{,} \PY{n}{linestyle}\PY{o}{=}\PY{l+s+s1}{\PYZsq{}}\PY{l+s+s1}{\PYZhy{}\PYZhy{}}\PY{l+s+s1}{\PYZsq{}}\PY{p}{)}
\end{Verbatim}


\begin{Verbatim}[commandchars=\\\{\}]
{\color{outcolor}Out[{\color{outcolor}62}]:} [<matplotlib.lines.Line2D at 0x7f95eb1940f0>]
\end{Verbatim}
            
    \begin{Verbatim}[commandchars=\\\{\}]
{\color{incolor}In [{\color{incolor}63}]:} \PY{c+c1}{\PYZsh{} Alternatively, styles can be spelled out as well (this is true for colors, etc.) }
         \PY{n}{plt}\PY{o}{.}\PY{n}{plot}\PY{p}{(}\PY{n}{x}\PY{p}{,} \PY{n}{y}\PY{p}{,} \PY{n}{linestyle}\PY{o}{=}\PY{l+s+s1}{\PYZsq{}}\PY{l+s+s1}{dashed}\PY{l+s+s1}{\PYZsq{}}\PY{p}{,} \PY{n}{color} \PY{o}{=} \PY{l+s+s1}{\PYZsq{}}\PY{l+s+s1}{orange}\PY{l+s+s1}{\PYZsq{}}\PY{p}{,} \PY{n}{linewidth} \PY{o}{=} \PY{l+m+mi}{8}\PY{p}{)}
\end{Verbatim}


\begin{Verbatim}[commandchars=\\\{\}]
{\color{outcolor}Out[{\color{outcolor}63}]:} [<matplotlib.lines.Line2D at 0x7f95eb024400>]
\end{Verbatim}
            
    \begin{center}
    \adjustimage{max size={0.9\linewidth}{0.9\paperheight}}{output_126_1.png}
    \end{center}
    { \hspace*{\fill} \\}
    
    \begin{Verbatim}[commandchars=\\\{\}]
{\color{incolor}In [{\color{incolor}64}]:} \PY{c+c1}{\PYZsh{} You will rarely work with variables to plot that are this simple. }
         \PY{c+c1}{\PYZsh{} Instead, it makes sense to learn how to define your x axis as continuously }
         \PY{c+c1}{\PYZsh{} increasing so that you can plot your variable of interest on y\PYZhy{}axis}
         
         \PY{c+c1}{\PYZsh{} Here we take advantage of numpy\PYZsq{}s functions linspace and sin.}
         
         \PY{k+kn}{import} \PY{n+nn}{numpy} \PY{k}{as} \PY{n+nn}{np}
         
         \PY{n}{x} \PY{o}{=} \PY{n}{np}\PY{o}{.}\PY{n}{linspace}\PY{p}{(}\PY{l+m+mi}{0}\PY{p}{,} \PY{l+m+mi}{10}\PY{p}{,} \PY{l+m+mi}{1000}\PY{p}{)}
         
         \PY{n}{plt}\PY{o}{.}\PY{n}{plot}\PY{p}{(}\PY{n}{x}\PY{p}{,} \PY{n}{np}\PY{o}{.}\PY{n}{sin}\PY{p}{(}\PY{n}{x}\PY{p}{)}\PY{p}{,} \PY{n}{c} \PY{o}{=} \PY{l+s+s1}{\PYZsq{}}\PY{l+s+s1}{magenta}\PY{l+s+s1}{\PYZsq{}}\PY{p}{,} \PY{n}{linewidth} \PY{o}{=} \PY{l+m+mf}{2.5}\PY{p}{)}
\end{Verbatim}


\begin{Verbatim}[commandchars=\\\{\}]
{\color{outcolor}Out[{\color{outcolor}64}]:} [<matplotlib.lines.Line2D at 0x7f95eaf989b0>]
\end{Verbatim}
            
    \begin{center}
    \adjustimage{max size={0.9\linewidth}{0.9\paperheight}}{output_127_1.png}
    \end{center}
    { \hspace*{\fill} \\}
    
    \begin{Verbatim}[commandchars=\\\{\}]
{\color{incolor}In [{\color{incolor}65}]:} \PY{c+c1}{\PYZsh{} It\PYZsq{}s likely you\PYZsq{}ll often want to look at a relationship between two variables. }
         \PY{c+c1}{\PYZsh{} You can do this simply by scattering them in matplotlib... }
         
         \PY{n}{x} \PY{o}{=} \PY{p}{[}\PY{l+m+mi}{1}\PY{p}{,} \PY{l+m+mi}{2}\PY{p}{,} \PY{l+m+mi}{3}\PY{p}{,} \PY{l+m+mi}{4}\PY{p}{,} \PY{l+m+mi}{5}\PY{p}{,} \PY{l+m+mi}{1}\PY{p}{,} \PY{l+m+mi}{2}\PY{p}{,} \PY{l+m+mi}{3}\PY{p}{,} \PY{l+m+mi}{4}\PY{p}{,} \PY{l+m+mi}{5}\PY{p}{,} \PY{l+m+mi}{1}\PY{p}{,} \PY{l+m+mi}{2}\PY{p}{,} \PY{l+m+mi}{3}\PY{p}{,} \PY{l+m+mi}{4}\PY{p}{,} \PY{l+m+mi}{5}\PY{p}{]}
         \PY{n}{y} \PY{o}{=} \PY{p}{[}\PY{l+m+mi}{2}\PY{p}{,} \PY{l+m+mi}{5}\PY{p}{,} \PY{l+m+mi}{6}\PY{p}{,} \PY{l+m+mi}{11}\PY{p}{,} \PY{l+m+mi}{14}\PY{p}{,} \PY{l+m+mi}{5}\PY{p}{,} \PY{l+m+mi}{2}\PY{p}{,} \PY{l+m+mi}{7}\PY{p}{,} \PY{l+m+mi}{8}\PY{p}{,} \PY{l+m+mi}{9}\PY{p}{,} \PY{l+m+mi}{1}\PY{p}{,} \PY{l+m+mi}{2}\PY{p}{,} \PY{l+m+mi}{3}\PY{p}{,} \PY{l+m+mi}{7}\PY{p}{,} \PY{l+m+mi}{3}\PY{p}{]}
         
         \PY{n}{plt}\PY{o}{.}\PY{n}{scatter}\PY{p}{(}\PY{n}{x}\PY{p}{,} \PY{n}{y}\PY{p}{,} \PY{n}{color} \PY{o}{=} \PY{l+s+s1}{\PYZsq{}}\PY{l+s+s1}{royalblue}\PY{l+s+s1}{\PYZsq{}}\PY{p}{)}
\end{Verbatim}


\begin{Verbatim}[commandchars=\\\{\}]
{\color{outcolor}Out[{\color{outcolor}65}]:} <matplotlib.collections.PathCollection at 0x7f95eaef5ef0>
\end{Verbatim}
            
    \begin{center}
    \adjustimage{max size={0.9\linewidth}{0.9\paperheight}}{output_128_1.png}
    \end{center}
    { \hspace*{\fill} \\}
    
    \begin{Verbatim}[commandchars=\\\{\}]
{\color{incolor}In [{\color{incolor}66}]:} \PY{c+c1}{\PYZsh{} Alternatively, some more powerful visualizations are available in seaborn. }
         \PY{c+c1}{\PYZsh{} Here, we plot the exact same as in the figure above, but denoting the line of best }
         \PY{c+c1}{\PYZsh{} fit for this data. }
         
         \PY{k+kn}{import} \PY{n+nn}{pandas} \PY{k}{as} \PY{n+nn}{pd}
         \PY{n}{dataframe} \PY{o}{=} \PY{n}{pd}\PY{o}{.}\PY{n}{DataFrame}\PY{p}{(}\PY{p}{\PYZob{}}\PY{l+s+s1}{\PYZsq{}}\PY{l+s+s1}{x}\PY{l+s+s1}{\PYZsq{}}\PY{p}{:} \PY{n}{x}\PY{p}{,} 
                                 \PY{l+s+s1}{\PYZsq{}}\PY{l+s+s1}{y}\PY{l+s+s1}{\PYZsq{}}\PY{p}{:} \PY{n}{y}\PY{p}{\PYZcb{}}\PY{p}{)}
         
         
         \PY{k+kn}{import} \PY{n+nn}{seaborn} \PY{k}{as} \PY{n+nn}{sns} 
         \PY{n}{g} \PY{o}{=} \PY{n}{sns}\PY{o}{.}\PY{n}{lmplot}\PY{p}{(}\PY{l+s+s1}{\PYZsq{}}\PY{l+s+s1}{x}\PY{l+s+s1}{\PYZsq{}}\PY{p}{,} \PY{l+s+s1}{\PYZsq{}}\PY{l+s+s1}{y}\PY{l+s+s1}{\PYZsq{}}\PY{p}{,} \PY{n}{data} \PY{o}{=} \PY{n}{dataframe}\PY{p}{)}
\end{Verbatim}


    \begin{center}
    \adjustimage{max size={0.9\linewidth}{0.9\paperheight}}{output_129_0.png}
    \end{center}
    { \hspace*{\fill} \\}
    
    \begin{Verbatim}[commandchars=\\\{\}]
{\color{incolor}In [{\color{incolor}67}]:} \PY{c+c1}{\PYZsh{} Want to know how related your variables are? }
         \PY{c+c1}{\PYZsh{} There are always multiple ways to do things in Python. Here is an example of }
         \PY{c+c1}{\PYZsh{} how even something as simple as a correlation between two variables can be calculated }
         \PY{c+c1}{\PYZsh{} using multiple packages. }
         
         \PY{n}{x} \PY{o}{=} \PY{p}{[}\PY{l+m+mi}{1}\PY{p}{,} \PY{l+m+mi}{2}\PY{p}{,} \PY{l+m+mi}{3}\PY{p}{,} \PY{l+m+mi}{4}\PY{p}{,} \PY{l+m+mi}{5}\PY{p}{,} \PY{l+m+mi}{1}\PY{p}{,} \PY{l+m+mi}{2}\PY{p}{,} \PY{l+m+mi}{3}\PY{p}{,} \PY{l+m+mi}{4}\PY{p}{,} \PY{l+m+mi}{5}\PY{p}{,} \PY{l+m+mi}{1}\PY{p}{,} \PY{l+m+mi}{2}\PY{p}{,} \PY{l+m+mi}{3}\PY{p}{,} \PY{l+m+mi}{4}\PY{p}{,} \PY{l+m+mi}{5}\PY{p}{]}
         \PY{n}{y} \PY{o}{=} \PY{p}{[}\PY{l+m+mi}{2}\PY{p}{,} \PY{l+m+mi}{5}\PY{p}{,} \PY{l+m+mi}{6}\PY{p}{,} \PY{l+m+mi}{11}\PY{p}{,} \PY{l+m+mi}{14}\PY{p}{,} \PY{l+m+mi}{5}\PY{p}{,} \PY{l+m+mi}{2}\PY{p}{,} \PY{l+m+mi}{7}\PY{p}{,} \PY{l+m+mi}{8}\PY{p}{,} \PY{l+m+mi}{9}\PY{p}{,} \PY{l+m+mi}{1}\PY{p}{,} \PY{l+m+mi}{2}\PY{p}{,} \PY{l+m+mi}{3}\PY{p}{,} \PY{l+m+mi}{7}\PY{p}{,} \PY{l+m+mi}{3}\PY{p}{]}
         
         \PY{n+nb}{print}\PY{p}{(}\PY{l+s+s1}{\PYZsq{}}\PY{l+s+s1}{Correlation using numpy: }\PY{l+s+s1}{\PYZsq{}}\PY{p}{,} \PY{n}{np}\PY{o}{.}\PY{n}{corrcoef}\PY{p}{(}\PY{n}{x}\PY{p}{,} \PY{n}{y}\PY{p}{)}\PY{p}{[}\PY{l+m+mi}{0}\PY{p}{]}\PY{p}{[}\PY{l+m+mi}{1}\PY{p}{]}\PY{p}{)}
         
         \PY{k+kn}{from} \PY{n+nn}{scipy}\PY{n+nn}{.}\PY{n+nn}{stats} \PY{k}{import} \PY{n}{linregress}
         \PY{n+nb}{print}\PY{p}{(}\PY{l+s+s1}{\PYZsq{}}\PY{l+s+s1}{Correlation using scipy: }\PY{l+s+s1}{\PYZsq{}}\PY{p}{,} \PY{n}{linregress}\PY{p}{(}\PY{n}{x}\PY{p}{,} \PY{n}{y}\PY{p}{)}\PY{p}{[}\PY{l+m+mi}{2}\PY{p}{]}\PY{p}{)}
\end{Verbatim}


    \begin{Verbatim}[commandchars=\\\{\}]
Correlation using numpy:  0.6923521761619424
Correlation using scipy:  0.6923521761619424

    \end{Verbatim}

    \section{Part 3: Outlook}\label{part-3-outlook}

    We're now at the end of our little taster session. Good job!

Most importantly: Don't worry if you feel a little bit overwhelmed. We
covered quite a lot of material that is usually taught over several
weeks. But this notebook is self-contained and will remain available
online. Feel free to revisit it whenever you feel like learning more
about Python or brushing up your skills!

Also, there is no need to remember every single nitty gritty bit of a
programming language. Even experts spend most of their time googling and
searching stackoverflow.com for solutions to their problems. That's one
of the great benefits of working with computers and being connected to
the internet! A solution is always only a few clicks away.

Below we've collected a few ressources that we think come in handy if
you'd like to learn more. Thanks a lot for your attention!

    \subsection{Resources}\label{resources}

    \paragraph{Learn coding!}\label{learn-coding}

\begin{enumerate}
\def\labelenumi{\arabic{enumi}.}
\tightlist
\item
  A complete course https://www.learnpython.org
\item
  Ditto https://www.w3schools.com/python/
\item
  Advanced Course https://automatetheboringstuff.com
\end{enumerate}

    \paragraph{How to install Python}\label{how-to-install-python}

\begin{enumerate}
\def\labelenumi{\arabic{enumi}.}
\tightlist
\item
  Just python https://www.codecademy.com/articles/install-python
\item
  The Jupyter notebook (the thing you've been working with today)
  https://jupyter.org/install
\item
  Anaconda (a collection of useful packages and other software for data
  scientists) https://www.anaconda.com/distribution/
\end{enumerate}

\paragraph{Text Editors}\label{text-editors}

Note: If you don't use notebooks, you'll need a text editor to writer
your python code. By all means, don't use Microsoft Word! You want
something that automatically highlights code (makes it easier to
distinguish variables, functions and the like), and provides
autocompletion (saves a lot of time!). Below are a few free and popular
choices: 1. atom editor https://atom.io/ 2. vscode
https://code.visualstudio.com/ 3. Sublime https://www.sublimetext.com/
All thes editors support multiple programming languages. Thus no need to
switch between programmes. I write my Python, Matlab, Javascript code
and documentation in Markdown or Latex in Atom. Saves so much time!

\paragraph{All in One Solutions}\label{all-in-one-solutions}

You can also use an all in one solution that provides you with an
editor, code interpreter (the thing that allows the computer to read and
execute python code) and file managers. A bit like the Matlab software
or R-Studio, which you might have heard of, but for python:

\begin{enumerate}
\def\labelenumi{\arabic{enumi}.}
\tightlist
\item
  Spyder (free) https://www.spyder-ide.org/
\item
  Pycharm (free basic and commercial pro version)
  https://www.jetbrains.com/pycharm/
\end{enumerate}

    \paragraph{Coding Challenges}\label{coding-challenges}

These challenges are a great way to test your understanding of computer
science concepts and improve your logical thinking / problem solving
skills. Also good to know: Almost every company uses these to assess
candidates for software engineering / data science positions 1.
Hackerrank https://www.hackerrank.com/ 2. Leetcode https://leetcode.com

    \paragraph{Python for Psychologists}\label{python-for-psychologists}

You can use python to run your own experiments, analyse behavioural,
eye-tracking and neuroimaging data and to make beautiful visualisations
of your results. Below are a few pointers
https://www.marsja.se/best-python-libraries-psychology/


    % Add a bibliography block to the postdoc
    
    
    
    \end{document}
